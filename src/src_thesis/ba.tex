%%%%%%%%%%%%%%%%%%%%%%%%%%%%%%%%%%%%%%%%%%%%%%%%%%%%%%%%%%%%%%%
%
% Beispiel für eine Bachelorarbeit mit FH FB 8 Titelblattstyles
%
%	Prof. Enning, 18.07.2013
%   Überarbeitet, 17.06.2014
%
%%%%%%%%%%%%%%%%%%%%%%%%%%%%%%%%%%%%%%%%%%%%%%%%%%%%%%%%%%%%%%%

%%%%%%%%%%%%%% Präambel %%%%%%%%%%%%%%%%%%%%%%%%%%%%%%%%%%%%%%%%
\documentclass [
% 	draft		% falls ohne Bilder gedruckt werden soll (Entwurf)
	]{scrbook}	% KOMA Klasse für Bücher
%
\usepackage{fhacmb}	% Style-File für Titelblatt. Ggf. bei Enning holen
%
% Einstellen der Optionen für die KOMA Klasse
\KOMAoptions{
	parskip=true,		% Absätze mit Abstand
	fontsize=12pt,		% Standardschriftgröße 
	toc=flat,		% Inhaltsverzeichnis ohne Einzüge
	twoside=false,		% Einseitig setzen
	numbers=nodotatend,	% Nummerierungen nicht mit Punkt abschließen
% die folgenden Optionen nehmen die entsprechende Dinge ins Inhaltsverzeichnis auf
% mit der bei texlive vorhandenen aktuellen Version von pdflatex funkioniert es nicht mehr
% (bekannter Bug)
%	toc=bibliography,	% Literaturverzeichnis ins Inhaltsverzeichnis
%	toc=listof,		% Abbildung- und Tabellenverzeichnis ins Inhaltsverzeichnis
%	toc=index,		% Stichwortverzeichnis ins Inhaltsverzeichnis
	}
%
%%%%%% Immer benötigte Packages
%
\usepackage[T1]{fontenc}		% sonst funktioniert die Silbentrennung bei Umlauten nicht
\usepackage[utf8]{inputenc}	% Eingabedekodierung. Ermöglicht Umlaute. Achtung: Unbedingt mit Betreuer
				% Verwendung der Umlaute-Eingabemethode absprechen. Im Zweifel \"O für Ö
\usepackage[ngerman]{babel}	% Silbentrennung und Sprachanpassung
\usepackage{blindtext}		% Blindtext
\usepackage[hidelinks]{hyperref}		% Sprungmarken, z.B. im Inhaltsverzeichnis auf Textpassagen
\usepackage{graphicx}		% Definiert o.a. \includegraphics
\usepackage{textcomp}		% Sonst funktioniert z.B. \texteuro nicht
\usepackage{scrlayer-scrpage}	% Package zum Definieren der Kopf- und Fußzeilen
\usepackage{amsmath}		% Muss sein
\usepackage{mathrsfs} 	% Weitere Mathematik-Symbole, z.B. Laplace-L
%
%%%%% Anpassung an Formatvorlagen des Fachbereichs
%
\usepackage{helvet}		% Serifenlose Schrift ähnlich Arial
\renewcommand{\familydefault}{\sfdefault}	% als Standardschrift serifenlose Schrift verwenden
%
\usepackage{geometry} 		% Ränder direkt einstellen
\geometry{a4paper, top=20mm, left=30mm, right=20mm, bottom=25mm} % nach Vorgabe
\linespread{1.25} 		% Zeilenabstand nach Vorgabe
%
\usepackage{chngcntr}		% Ändert Verhalten von Countern
\counterwithout{figure}{section}	% Figure-Nummerierung nicht bei section-Wechsel zurücksetzen
\renewcommand{\thefigure}{\textbf\thechapter-\arabic{figure}}	% im Stil 3-2
%
%%%% für das Erzeugen von Grafiken mit Zeichenbefehlen
%
\usepackage{tikz}		% Grundpaket
\usetikzlibrary{shapes,arrows}	% einige Symbolpackages
\usepackage{tikz-cd}		% einige Symbolpackages
%
%%%%%% Gegebenenfalls nützliche Zusatzpackages
%
%\usepackage{blindtext}		% Blindtexte zum Ausprobieren von Formatierungen
%\usepackage{color}		% Schriftfarben
\usepackage{colortbl}		% für die Hintergrundfarbe von Tabellen
%\usepackage{gensymb}		% Definiert Formelzeichen, die im math und im text-Modus gleich aussehen
%\usepackage{wrapfig}		% Definiert die wrapfigure-Umgebung (Bild am Rand von Text umflossen)
%\usepackage{pdfpages}		% Einbinden von pdf-Seiten
%
%\usepackage{booktabs} 		% Schöne Tabellen
%\usepackage{tabu}	 	% Sehr einfach Tabellen gestalten
%\usepackage{array} 		% Erweiterung der Tabellenumgebung, neue Spaltentypen
\usepackage{paralist}		% Weitere Nummeriungsoptionen, z.b. alphabetisch für enumerate/itemize
%\usepackage{verbatim}		% Verbesserte verbatim-Umgebung (z.b. Programm-Listings)
%\usepackage{subfig}		% Unterfigures mit eigenen Bildunterschriften
%
%%%%%% Sammelsurium
%
%\renewcommand{\labelitemii}{$\circ$} % Bullets für itemize-Listen
%
%%%%%%%%%%  Angaben für Titelseite %%%%%%%%%%%%%%%%%%%%%%
%
% Angaben für Titelseite
\arbeitstyp{Bachelorarbeit}
\fachbereich{[FB5] Elektrotechnik und Informationstechnik}
\studiengang{Informatik}
\vertiefung{}
\titel{Integration eines eingebetteten Systems in eine Cloud-Infrastruktur am Beispiel eines autonomen Spielfelds}
\autor{Marcel Ochsendorf}
\matrnr{3120232}
\betreuer{Prof. Dr.-Ing. Thomas Dey}
\cobetreuer{-- TBA --}
\datum{25. Mai 2021}
\dank{}

%
%%%%%%%%%%%%%%%%%%%%%%%%%%%%%%%%%%%%%%%%%%%%%%%%%%%%%%%%%%%
%
\begin{document}
% Einige Anpassungen müssen nach \begin{document} stehen !!
\renewcaptionname{ngerman}{\figurename}{\textbf Bild} 	% Bild ... statt Abbildung ... 
\renewcaptionname{ngerman}{\contentsname}{Inhalt}% Inhalt statt Inhaltsverzeichnis
%%%%%%%%%%%%%%%%%%%%%%%%%%%%%%%%%%%%%%%%%%%%%%%%%%%%%%%%%%%%
% Titel im FH Style
%%%%%%%%%%%%%%%%%%%%%%%%%%%%%%%%%%%%%%%%%%%%%%%%%%%%%%%%%%%%
\fhacmbtitle{\includegraphics[height=4cm]{fh_logo}}{5pt}{5pt}

%%%%%%%%%%%%%%%%%%%%%%%%%%%%%%%%%%%%%%%%%%%%%%%%%%%%%%%%%%%%
% Erklärung / Geheimhaltung
%%%%%%%%%%%%%%%%%%%%%%%%%%%%%%%%%%%%%%%%%%%%%%%%%%%%%%%%%%%%

%\frontmatter 	% Wenn der Hauptteil mit Seite 1 beginnen soll
\pagestyle{plain}
% \input{erklaerung}

%%%%%%%%%%%%%%%%%%%%%%%%%%%%%%%%%%%%%%%%%%%%%%%%%%%%%%%%%%%%
% Inhaltsverzeichnis
%%%%%%%%%%%%%%%%%%%%%%%%%%%%%%%%%%%%%%%%%%%%%%%%%%%%%%%%%%%%

\tableofcontents


%\mainmatter	% Wenn der Hauptteil mit Seite 1 beginnen soll
\pagestyle{scrheadings}
%%%%%%%%%%%%%%% Anpassung des Seitenstils an FH-Layoutvorschrift %%%%%%%%%%%%
\renewcommand{\chaptermark}[1]{\markboth{\thechapter\hspace{1cm}#1}{}}	% Kapitel für Headerzeile neu definieren (ohne Nummer)
\chead{}		% Header Mitte löschen
\ihead{\leftmark}	% Kapitelbezeichnung links setzen 
\renewcommand{\headfont}{\bfseries}	% Bold-Font für Headerzeile verwenden
\setheadsepline{0.5pt}

% \input{aufgabenstellung}
\chapter{Einführung} 
\section{Was ganz Allgemeines}
\subsection{Welche Dateien braucht man wirklich?}
Ein Compilerlauf erzeugt eine Unmenge Dateien. Die meisten integrierten Tex-Umgebungen bieten einen Befehl zum Aufräumen, der alles löscht, was nicht zu Ihrem Projekt gehört.

Folgende Dateien dürfen nie gelöscht werden

\begin{tabu}{|X[0.5] | X[3] |}
\hline
\rowfont{\bfseries} Endung & Inhalt \\
\hline
tex & Ihre Tex-Quelldateien\\
bib & Ihre Literaturdatenbank(en)\\
\hline
\end{tabu}
und natürlich alles an Material (Bilder etc.), was Sie einbinden. Fast alle anderen Dateien dürfen (und sollten gelegentlich) entsorgt werden, insbesondere

\begin{tabu}{|X[0.5] | X[3] |}
\hline
\rowfont{\bfseries} Endung & Inhalt \\
\hline
aux & Hilfsdateien\\
toc & Erzeugte Verzeichnisse\\
lot & \\
lof & \\
out & \\
blg & Log-Datei des Bibtex-Laufs \\
log & Log-Datei des pdflatex-Laufs\\
\hline
\end{tabu}


\subsection{Was ist eigentlich Blindtext?}

\blindtext
\subsection{Wie macht man Aufzählungen?}
Zum Beispiel als
\begin{itemize}
\item einfache
\item unnummerierte
\item Liste
\end{itemize}

oder auch als numerierte Liste mit
\begin{enumerate}
\item Erstens
\item Zweitens 
\item Drittens
\end{enumerate}

Sie können auch
\begin{enumerate}[a)]
\item ganz
\item anders 
\item nummerieren
\end{enumerate}

\section{Tabelle}
\subsection{Ein Buch mit sieben Siegeln}

Es gibt eine unüberschaubare Menge an Styles für Tabellen. Den einfachsten Einstieg erhalten Sie, wenn Sie nicht die tabular-Umgebung, sondern die tabu-Umgebung verwenden. Hier ein Beispiel, Tabelle \ref{tabelle1}.

\begin{table}
\caption{Tabellen beschriftet man \underline{oberhalb}}
\label{tabelle1}
\tabulinesep=3pt  % ein bisschen Platz schaffen
\begin{tabu} to 0.8\textwidth {| X[0.5c] | X[2r] | X[3l]|}
\hline
\rowfont{\bfseries} Nr & Aktion & Formel\\ 
\hline
1 & Bachelorarbeit schreiben &
$G(s) = \frac{1}{(s-s_p)^n}$
\\\hline
2 & Heiraten &
$G(s) = \frac{s\omega_0^2}{s^2+2D\omega_0 s + \omega_0^2}$ 
\\\hline
3 & Baum pflanzen &
$F(s) = \mathscr L \{f (t)\} = \int\limits_0^{\infty} f (t) \, e^{-s t} \text{d}t$
\\
\hline
\end{tabu}
\end{table}

\subsection{Verdammt nochmal, ich will selbst entscheiden}

... wohin meine Tabelle oder mein Bild kommt.

table und figure sind so genannte Float-Umgebungen, d.~h. Latex entscheidet, wo sie hinkommen und geht davon aus, dass es Ihnen egal ist, weil Sie die Objekte mit \textbackslash caption beschriften und über \textbackslash label und \textbackslash ref dorthin verweisen.

Wenn Sie das mögen (und damit sind Sie nicht alleine), verwenden Sie stattdessen andere Umgebungen, wie z.B. par oder center. Um dennoch eine Bild- bzw. Tabellennummer zu erhalten, benutzen Sie statt \textbackslash caption den Befehl \textbackslash captionof. Sie Beispiel Tabelle \ref{tabelle2} \underline{direkt nach diesem Absatz}.

\begin{center}
\captionof{table}{Tabellen beschriftet man \underline{oberhalb}}
\label{tabelle2}
\begin{tabu} to 0.8\textwidth {| X[0.5c] | X[2r] | X[3l]|}
\hline
\rowfont{\bfseries} Nr & Aktion & Formel\\ 
\hline
1 & Bachelorarbeit schreiben &
$G(s) = \frac{1}{(s-s_p)^n}$
\\\hline
2 & Heiraten &
$G(s) = \frac{s\omega_0^2}{s^2+2D\omega_0 s + \omega_0^2}$ 
\\\hline
3 & Baum pflanzen &
$F(s) = \L \{f (t)\} = \int\limits_0^{\infty} f (t) \, e^{-s t} \text{d}t$
\\
\hline
\end{tabu}
\end{center}

\section{Verwendung von Literatur}

Um das korrekte Zitieren kümmert sich bibtex. Legen Sie eine Datei (Datenbank) mit Ihrer Literatur als Datei \verb+[name].bib+ an. Darin nehmen Sie für jede Referenz eine Quellenangabe auf.

Tipp: Die meisten Literaturdatenbanken bieten die Möglichkeit, bibtex-Datensätze zu exportieren.

Wenn Sie auf Quellen, wie z.B. diese Webreferenz \cite{webreferenz} oder auch dieses Buch \cite{Murrenhoff} verweisen, erscheinen die entsprechenden Titel perfekt formatiert im Literaturverzeichnis.

Damit alle Referenzen stimmen, muss man ggf. pdflatex und bibtex mehrfach aufrufen. Sicher ist die folgende Reihenfolge
\begin{enumerate}
\item pdflatex. Erzeugt die *.aux-Datei, die bibtex benötigt
\item bibtex. Erzeugt Dateien, die pdflatex einbindet
\item pdflatex
\item pdflatex (Ggf. auf Fehlermeldungen schauen. Nach dem zweiten Lauf sollte alles okay sein)
\end{enumerate}



\subsection{Einfügen von Bildern}

Der richtige Tex-perte tex't auch seine Bilder, z. B. so wie in Bild \ref{bild2}

% Define block styles
	\tikzstyle{Baustein} = [rectangle, draw, 
    text width=3.5cm, text centered, rounded corners, minimum height=1cm]
	\tikzstyle{Baustein2} = [rectangle, draw, 
    text width=7cm, text centered, rounded corners, minimum height=1cm]
	\tikzstyle{Baustein3} = [rectangle, draw, 
    text width=9cm, text centered, rounded corners, minimum height=1cm]
	\tikzstyle{Baustein4} = [rectangle, draw, 
    text width=3.5cm, text centered, rounded corners, minimum height=4cm]
	 
	\tikzstyle{line} = [draw, -latex']
	\tikzstyle{line2} = [draw]

\begin{figure}
\begin{center}
\begin{tikzpicture}[node distance = 1.5cm, auto]
    % Place nodes
 	   \node [Baustein] (M) {Name};
 	   \node [Baustein, below of=M] (W) {Name};
 	   \node [Baustein, below of=W] (F) {Name};
  	  \node [Baustein, below of=F] (P) {Name};
  	  \node [Baustein2, below of=P] (S) {Name};
  	\coordinate (D) at (-4,-6);

    % Draw edges

   	 \path [line](S)--(P);
	\path [line](P) -- (F);
	\path [line](F) -- (W);
	\path [line](W)-- (M);
	\path[line2] (M)-|(D);
  	 \path[line] (D)--(S); 
\end{tikzpicture}
\caption{figure}{Ein mit tikz gezeichnetes Flussdiagramm}
\label{bild2}
\end{center}
\end{figure}
Bilder bindet man wie folgt ein. Siehe Beispiel Bild \ref{bild1}

\begin{figure}[h]
\centering
\includegraphics[width=0.1\textwidth]{fh_logo.png}
\caption{Logo der FH Aachen}
\label{bild1}
\end{figure}


\blindtext



% \input{kapitel2}

%% Verschiedene Versionen, nach DIN 1505 zu zitieren
\bibliographystyle{plaindin}
%\bibliographystyle{natdin}
%\bibliographystyle{alphadin}
%\bibliographystyle{unsrtdin}

% Die DIN 1505 Styles müssen ggf. nachinstalliert werden. Bei Texlive
% heisst das Paket texlive-bibtex-extra

% Hier muss noch aufgeräumt werden. Nennung von URLs verbesserungsbedürftig. Ggf auf biblatex (statt bibtex) umsteigen
\bibliography{ba}

\listoffigures
\listoftables

\appendix
% \input{anhang}

\end{document}