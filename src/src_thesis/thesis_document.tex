\hypertarget{einleitung}{%
\section{Einleitung}\label{einleitung}}

\hypertarget{motivation}{%
\subsection{Motivation}\label{motivation}}

\begin{itemize}
\tightlist
\item
  Beginn: Er zieht die Aufmerksamkeit des Lesers durch die Schilderung
  des Ereignisses auf sich, das zu dem Problem geführt hat.
\item
  Hintergrundinformationen (Herstellung des Kontexts): Gehe tiefer auf
  das Ereignis ein, indem du mehr Informationen über es vermittelst und
  dabei auch den Rahmen deiner Forschung skizzierst.
\item
  Brücke zur Problemstellung: Erläutere, inwiefern es sich hierbei um
  ein Problem handelt, und schlage somit die Brücke zur Problemstellung,
  die deiner Untersuchung zu Grunde liegt.
\end{itemize}

\hypertarget{zielsetzung}{%
\subsection{Zielsetzung}\label{zielsetzung}}

Das Ziel dieser Arbeit ist es, einen autonomen Schach-Tisch, welcher in
der Lage ist Schachfiguren autonom zu bewegen und auf
Benutzerinteraktion zu reagieren.

Der Schwerpunkt liegt dabei insbesondere auf der Programmierung des
eingebettenen Systems. Dieses besteht zum einem, aus der
Positionserkennung und Steuerung der Hardwarekomponenten (Schachfiguren)
und zum anderen aus der Kommunikation zwischen dem Tisch selbst und
einem in einer Cloud befindlichen Server.

Mittels der Programmierung werden diverse Technologien von verschiedenen
Einzelsystemen zu einem Gesamtprodukt zusammengesetzt. Zu diesen
Einzelsystemen gehören:

\begin{itemize}
\tightlist
\item
  Programmierung der Motorsteuerung, HMI (zB. Qt oder simple Buttons),
  NFC Tag erkennung
\item
  Programmierung eines Wrappers für die Kommunikation mit der Cloud
  (AWS)
\item
  Statemaschiene und Implementierung der Spielflusssteuerung
\item
  Backend mit Datenbankanbindung zwischen Server und Embedded-System
\item
  Verwendung eines CI/CD Systems zum automatisierten bauen der
  Linux-Images für das Embedded-System
\end{itemize}

\hypertarget{aufbau-der-arbeit}{%
\subsection{Aufbau der Arbeit}\label{aufbau-der-arbeit}}

beleuchtung existierender ansätze \&\& festlegung zu erwartener
Features, Kapitel x+1 zusammenführung in die DK HW,Kaptiel x+4 test und
fazit,demonstration und validierung der funktionsfähigkeit

\hypertarget{analyse-bestehender-systeme-und-machbarkeitsanalyse}{%
\section{Analyse bestehender Systeme und
Machbarkeitsanalyse}\label{analyse-bestehender-systeme-und-machbarkeitsanalyse}}

\hypertarget{existierende-systeme-im-vergleich}{%
\subsection{Existierende Systeme im
Vergleich}\label{existierende-systeme-im-vergleich}}

\hypertarget{kommerzielle-produkte}{%
\subsubsection{Kommerzielle Produkte}\label{kommerzielle-produkte}}

\begin{longtable}[]{@{}lllll@{}}
\caption{Auflistung kommerzieller autonomer Schachtische}\tabularnewline
\toprule
\begin{minipage}[b]{0.18\columnwidth}\raggedright
\strut
\end{minipage} & \begin{minipage}[b]{0.18\columnwidth}\raggedright
Square Off - Kingdom \cite{squareoffkingdom}\strut
\end{minipage} & \begin{minipage}[b]{0.22\columnwidth}\raggedright
Square Off - Grand Kingdom \cite{squareoffgrand}\strut
\end{minipage} & \begin{minipage}[b]{0.15\columnwidth}\raggedright
DGT Smart Board \cite{dtgsmartboard}\strut
\end{minipage} & \begin{minipage}[b]{0.13\columnwidth}\raggedright
DGT Bluetooth Wenge \cite{dtgble}\strut
\end{minipage}\tabularnewline
\midrule
\endfirsthead
\toprule
\begin{minipage}[b]{0.18\columnwidth}\raggedright
\strut
\end{minipage} & \begin{minipage}[b]{0.18\columnwidth}\raggedright
Square Off - Kingdom \cite{squareoffkingdom}\strut
\end{minipage} & \begin{minipage}[b]{0.22\columnwidth}\raggedright
Square Off - Grand Kingdom \cite{squareoffgrand}\strut
\end{minipage} & \begin{minipage}[b]{0.15\columnwidth}\raggedright
DGT Smart Board \cite{dtgsmartboard}\strut
\end{minipage} & \begin{minipage}[b]{0.13\columnwidth}\raggedright
DGT Bluetooth Wenge \cite{dtgble}\strut
\end{minipage}\tabularnewline
\midrule
\endhead
\begin{minipage}[t]{0.18\columnwidth}\raggedright
Erkennung Figurstellung\strut
\end{minipage} & \begin{minipage}[t]{0.18\columnwidth}\raggedright
nein (Manuell per Ausgangsposition)\strut
\end{minipage} & \begin{minipage}[t]{0.22\columnwidth}\raggedright
nein (Manuell per Ausgangsposition)\strut
\end{minipage} & \begin{minipage}[t]{0.15\columnwidth}\raggedright
ja\strut
\end{minipage} & \begin{minipage}[t]{0.13\columnwidth}\raggedright
ja\strut
\end{minipage}\tabularnewline
\begin{minipage}[t]{0.18\columnwidth}\raggedright
Abmessungen (LxBxH)\strut
\end{minipage} & \begin{minipage}[t]{0.18\columnwidth}\raggedright
486mm x 486mm x 75mm\strut
\end{minipage} & \begin{minipage}[t]{0.22\columnwidth}\raggedright
671mm x 486mm x 75mm\strut
\end{minipage} & \begin{minipage}[t]{0.15\columnwidth}\raggedright
540mm x 540mm x 20mm\strut
\end{minipage} & \begin{minipage}[t]{0.13\columnwidth}\raggedright
540mm x 540mm x 20mm\strut
\end{minipage}\tabularnewline
\begin{minipage}[t]{0.18\columnwidth}\raggedright
Konnektivität\strut
\end{minipage} & \begin{minipage}[t]{0.18\columnwidth}\raggedright
Bluetooth\strut
\end{minipage} & \begin{minipage}[t]{0.22\columnwidth}\raggedright
Bluetooth\strut
\end{minipage} & \begin{minipage}[t]{0.15\columnwidth}\raggedright
Seriell\strut
\end{minipage} & \begin{minipage}[t]{0.13\columnwidth}\raggedright
Bluetooth\strut
\end{minipage}\tabularnewline
\begin{minipage}[t]{0.18\columnwidth}\raggedright
Automatisches Bewegen der Figuren\strut
\end{minipage} & \begin{minipage}[t]{0.18\columnwidth}\raggedright
ja\strut
\end{minipage} & \begin{minipage}[t]{0.22\columnwidth}\raggedright
ja\strut
\end{minipage} & \begin{minipage}[t]{0.15\columnwidth}\raggedright
nein\strut
\end{minipage} & \begin{minipage}[t]{0.13\columnwidth}\raggedright
nein\strut
\end{minipage}\tabularnewline
\begin{minipage}[t]{0.18\columnwidth}\raggedright
Spiel Livestream\strut
\end{minipage} & \begin{minipage}[t]{0.18\columnwidth}\raggedright
ja\strut
\end{minipage} & \begin{minipage}[t]{0.22\columnwidth}\raggedright
ja\strut
\end{minipage} & \begin{minipage}[t]{0.15\columnwidth}\raggedright
ja\strut
\end{minipage} & \begin{minipage}[t]{0.13\columnwidth}\raggedright
ja\strut
\end{minipage}\tabularnewline
\begin{minipage}[t]{0.18\columnwidth}\raggedright
Cloud anbindung (online Spiele)\strut
\end{minipage} & \begin{minipage}[t]{0.18\columnwidth}\raggedright
ja (Mobiltelefon + App)\strut
\end{minipage} & \begin{minipage}[t]{0.22\columnwidth}\raggedright
ja (Mobiltelefon + App)\strut
\end{minipage} & \begin{minipage}[t]{0.15\columnwidth}\raggedright
ja (PC + App)\strut
\end{minipage} & \begin{minipage}[t]{0.13\columnwidth}\raggedright
ja (PC + App)\strut
\end{minipage}\tabularnewline
\begin{minipage}[t]{0.18\columnwidth}\raggedright
Parkposition für ausgeschiedene Figuren\strut
\end{minipage} & \begin{minipage}[t]{0.18\columnwidth}\raggedright
nein\strut
\end{minipage} & \begin{minipage}[t]{0.22\columnwidth}\raggedright
ja\strut
\end{minipage} & \begin{minipage}[t]{0.15\columnwidth}\raggedright
nein\strut
\end{minipage} & \begin{minipage}[t]{0.13\columnwidth}\raggedright
nein\strut
\end{minipage}\tabularnewline
\begin{minipage}[t]{0.18\columnwidth}\raggedright
Stand-Alone Funktionalität\strut
\end{minipage} & \begin{minipage}[t]{0.18\columnwidth}\raggedright
nein (Mobiltelefon erforderlich)\strut
\end{minipage} & \begin{minipage}[t]{0.22\columnwidth}\raggedright
nein (Mobiltelefon erforderlich)\strut
\end{minipage} & \begin{minipage}[t]{0.15\columnwidth}\raggedright
nein (PC erforderlich)\strut
\end{minipage} & \begin{minipage}[t]{0.13\columnwidth}\raggedright
nein (PC erforderlich)\strut
\end{minipage}\tabularnewline
\begin{minipage}[t]{0.18\columnwidth}\raggedright
Besonderheiten\strut
\end{minipage} & \begin{minipage}[t]{0.18\columnwidth}\raggedright
Akku für 30 Spiele\strut
\end{minipage} & \begin{minipage}[t]{0.22\columnwidth}\raggedright
Akku für 15 Spiele\strut
\end{minipage} & \begin{minipage}[t]{0.15\columnwidth}\raggedright
-\strut
\end{minipage} & \begin{minipage}[t]{0.13\columnwidth}\raggedright
-\strut
\end{minipage}\tabularnewline
\bottomrule
\end{longtable}

Bei den DGT Schachbrettern ist zu beachten, dass diese die Schachfiguren
nicht autonom bewegen können. Sie wurden jedoch in die Liste
aufgenommen, da diese einen Teil der Funktionalitäten der Square Off
Schachbrettern abdecken und lediglich die automatische Bewegung der
Schachfiguren fehlt. Die DGT-Bretter können die Position der Figuren
erkennen und ermöglichen so auch Spiele über das Internet; diese können
sie auch als Livestream anbieten. Bei Schachtunieren werden diese für
die Übertragung der Partien sowie die Aufzeichnung der Spielzüge
verwendet und bieten Support für den Anschluss von weiterer Peripherien
wie z.B. Schachuhren.

Somit gibt es zum Zeitpunkt der Recherche nur einen Hersteller von
autonomen Schachbrettern, welcher auch die Figuren bewegen kann.sdfger

wdad

\hypertarget{open-source-projekte}{%
\subsubsection{Open-Source Projekte}\label{open-source-projekte}}

Bei allen Open-Source Projekten wurden die Features anhand der
Beschreibung und der aktuellen Software extrahiert. Besonders bei
work-in-progress Projekten können sich die Features noch verändern und
so weitere Funktionalitäten hinzugefügt werden.

Zusätzlich zu den genannten Projekten sind weitere derartige Projekte
verfügbar; in der Tabelle wurde nur jende aufgelistet, welche sich von
anderen Projekten in mindestens einem Feature unterscheiden.

Auch existieren weitere Abwandlungen von autonomen Schachbrettern, bei
welchem die Figuren von oberhalb des Spielbretts gegriffen bzw. bewegt
werden. In einigen Projekten wird dies mittels eines Industrie-Roboters
\cite{actprojectrobot} oder eines modifizierten
3D-Druckers\cite{atcproject3dprinter} realisiert. Diese wurden hier
aufgrund der Mechanik, welche über dem Spielbrett montiert werden muss,
nicht berücksichtigt.

\begin{longtable}[]{@{}llll@{}}
\caption{Auflistung von Open-Source Schachtisch
Projekten}\tabularnewline
\toprule
\begin{minipage}[b]{0.20\columnwidth}\raggedright
\strut
\end{minipage} & \begin{minipage}[b]{0.24\columnwidth}\raggedright
Automated Chess Board (Michael Guerero) \cite{actproject1}\strut
\end{minipage} & \begin{minipage}[b]{0.26\columnwidth}\raggedright
Automated Chess Board (Akash Ravichandran) \cite{actproject2}\strut
\end{minipage} & \begin{minipage}[b]{0.19\columnwidth}\raggedright
DIY Super Smart Chessboard \cite{actproject3}\strut
\end{minipage}\tabularnewline
\midrule
\endfirsthead
\toprule
\begin{minipage}[b]{0.20\columnwidth}\raggedright
\strut
\end{minipage} & \begin{minipage}[b]{0.24\columnwidth}\raggedright
Automated Chess Board (Michael Guerero) \cite{actproject1}\strut
\end{minipage} & \begin{minipage}[b]{0.26\columnwidth}\raggedright
Automated Chess Board (Akash Ravichandran) \cite{actproject2}\strut
\end{minipage} & \begin{minipage}[b]{0.19\columnwidth}\raggedright
DIY Super Smart Chessboard \cite{actproject3}\strut
\end{minipage}\tabularnewline
\midrule
\endhead
\begin{minipage}[t]{0.20\columnwidth}\raggedright
Erkennung Figurstellung\strut
\end{minipage} & \begin{minipage}[t]{0.24\columnwidth}\raggedright
nein (Manuell per Ausgangsposition)\strut
\end{minipage} & \begin{minipage}[t]{0.26\columnwidth}\raggedright
ja (Kamera / OpenCV)\strut
\end{minipage} & \begin{minipage}[t]{0.19\columnwidth}\raggedright
nein\strut
\end{minipage}\tabularnewline
\begin{minipage}[t]{0.20\columnwidth}\raggedright
Abmessungen (LxBxH)\strut
\end{minipage} & \begin{minipage}[t]{0.24\columnwidth}\raggedright
keine Angabe\strut
\end{minipage} & \begin{minipage}[t]{0.26\columnwidth}\raggedright
keine Angabe\strut
\end{minipage} & \begin{minipage}[t]{0.19\columnwidth}\raggedright
450mm x 300mm x 50mm\strut
\end{minipage}\tabularnewline
\begin{minipage}[t]{0.20\columnwidth}\raggedright
Konnektivität\strut
\end{minipage} & \begin{minipage}[t]{0.24\columnwidth}\raggedright
\gls{usb}\strut
\end{minipage} & \begin{minipage}[t]{0.26\columnwidth}\raggedright
\gls{wlan}\strut
\end{minipage} & \begin{minipage}[t]{0.19\columnwidth}\raggedright
\gls{wlan}\strut
\end{minipage}\tabularnewline
\begin{minipage}[t]{0.20\columnwidth}\raggedright
Automatisches Bewegen der Figuren\strut
\end{minipage} & \begin{minipage}[t]{0.24\columnwidth}\raggedright
ja\strut
\end{minipage} & \begin{minipage}[t]{0.26\columnwidth}\raggedright
ja\strut
\end{minipage} & \begin{minipage}[t]{0.19\columnwidth}\raggedright
nein\strut
\end{minipage}\tabularnewline
\begin{minipage}[t]{0.20\columnwidth}\raggedright
Spiel Livestream\strut
\end{minipage} & \begin{minipage}[t]{0.24\columnwidth}\raggedright
nein\strut
\end{minipage} & \begin{minipage}[t]{0.26\columnwidth}\raggedright
nein\strut
\end{minipage} & \begin{minipage}[t]{0.19\columnwidth}\raggedright
nein\strut
\end{minipage}\tabularnewline
\begin{minipage}[t]{0.20\columnwidth}\raggedright
Cloud anbindung (online Spiele)\strut
\end{minipage} & \begin{minipage}[t]{0.24\columnwidth}\raggedright
nein\strut
\end{minipage} & \begin{minipage}[t]{0.26\columnwidth}\raggedright
nein\strut
\end{minipage} & \begin{minipage}[t]{0.19\columnwidth}\raggedright
ja\strut
\end{minipage}\tabularnewline
\begin{minipage}[t]{0.20\columnwidth}\raggedright
Parkposition für ausgeschiedene Figuren\strut
\end{minipage} & \begin{minipage}[t]{0.24\columnwidth}\raggedright
nein\strut
\end{minipage} & \begin{minipage}[t]{0.26\columnwidth}\raggedright
nein\strut
\end{minipage} & \begin{minipage}[t]{0.19\columnwidth}\raggedright
nein\strut
\end{minipage}\tabularnewline
\begin{minipage}[t]{0.20\columnwidth}\raggedright
Stand-Alone Funktionalität\strut
\end{minipage} & \begin{minipage}[t]{0.24\columnwidth}\raggedright
nein (PC erforderlich)\strut
\end{minipage} & \begin{minipage}[t]{0.26\columnwidth}\raggedright
ja\strut
\end{minipage} & \begin{minipage}[t]{0.19\columnwidth}\raggedright
ja\strut
\end{minipage}\tabularnewline
\begin{minipage}[t]{0.20\columnwidth}\raggedright
Besonderheiten\strut
\end{minipage} & \begin{minipage}[t]{0.24\columnwidth}\raggedright
-\strut
\end{minipage} & \begin{minipage}[t]{0.26\columnwidth}\raggedright
Sprachsteuerung (Amazon Alexa)\strut
\end{minipage} & \begin{minipage}[t]{0.19\columnwidth}\raggedright
Zuganzeige über LED Matrix\strut
\end{minipage}\tabularnewline
\begin{minipage}[t]{0.20\columnwidth}\raggedright
Lizenz\strut
\end{minipage} & \begin{minipage}[t]{0.24\columnwidth}\raggedright
\gls{gpl} 3+\strut
\end{minipage} & \begin{minipage}[t]{0.26\columnwidth}\raggedright
\gls{gpl}\strut
\end{minipage} & \begin{minipage}[t]{0.19\columnwidth}\raggedright
-\strut
\end{minipage}\tabularnewline
\bottomrule
\end{longtable}

In den bestehenden Projekten ist zu erkennen, dass ein autonomer
Schachtisch sehr einfach und mit einfachsten Mittel konstruiert werden
kann. Hierbei fehlen in der Regel einige Features, wie das automatische
Erkennen von Figuren oder das Spielen über das Internet.

Einige Projekte setzten dabei auf eingebettete Systeme, welche direkt im
Schachtisch montiert sind, andere hingegen nutzen einen externen PC,
welcher die Steuerbefehle an die Elektronik sendet.

Bei der Konstruktion der Mechanik und der Methode mit welcher die
Figuren über das Feld bewegt werden ähneln sich jedoch die meisten
dieser Projekte. Hier wird in der Regel eine einfache X und Y-Achse
verwendet, welche von zwei Schrittmotoren bewegt werden. Die
Schachfiguren werden dabei mittels eines Elektromagneten über die
Oberseite gezogen. Hierbei ist ein Magnet oder eine kleine Metallplatte
in den Fuß der Figuren eingelassen worden.

Die Erkennung der Schachfiguren ist augenscheinlich die schwierigste
Aufgabe. Hier wurde in der Mehrzahl der Projekte eine Kamera im
Zusammenspiel mit einer auf OpenCV basierenden Figur-Erkennung
verwendet. Diese Variante ist je nach Implementierung des
Vision-Algorithmus fehleranfälliger bei sich ändernden
Lichtverhältnissen, auch muss die Kamera oberhalb der Schachfiguren
platziert werden, wenn kein transparentes Schachfeld verwendet werden
soll.

Eine andere Alternative ist die Verwendung einer Matrix aus
Reed-Schaltern oder Hallsensoren. Diese werden in einer 8x8 Matrix
Konfiguration unterhalb der Platte montiert und reagieren auf die
Magnete in den Figuren. So ist es möglich zu erkennen, welches der
Schachfelder belegt ist, jedoch nicht konkret von welchem Figurtypen.
Dieses Problem wird durch eine definierte Ausgangsstellung beim
Spielstart gelöst. Nach jedem Zug durch den Spieler und der dadurch
resultierenden Änderungen in der Figurpositionen in der Matrix können
die neuen Figurstellungen berechnet werden.

\hypertarget{user-experience}{%
\subsection{User Experience}\label{user-experience}}

Ein wichtiger Aspekt bei diesem Projekt stellt die User-Experience dar.
Diese beschreibt die Ergonomie der Mensch-Maschine-Interaktion und wird
durch die DIN 9241\cite{din9241} beschrieben. Hierbei geht es primär
um das Erlebnis, welches der Benutzer bei dem Verwenden eines Produktes
erlebt und welche Erwartungen der Benutzer an die Verwendung des
Produktes hat.

Bei dem autonomen Schachtisch, soll der Benutzer eine ähnlich einfache
Erfahrung erleben, wie bei einer Schachpartie mit einem menschlichen
Gegenspieler. Der Benutzer soll direkt nach dem Einschalten des Tisches
und dem Aufstellen der Figuren in der Lage sein, mit dem Spiel beginnen
zu können. Dies soll wie ein reguläres Schachspiel ablaufen; der Spieler
vor dem Tisch soll die Figuren mit der Hand bewegen können und der Tisch
soll den Gegenspieler darstellen. Dieser bewegt die Figuren der
Gegenseite.

Nach Beendigung einer Partie, soll das Spielbrett wieder in die
Ausgangssituation gebracht werden; dies kann zum einem vom Tisch selbst
oder vom Benutzer manuell geschehen. Danach ist der Tisch für die
nächste Partie bereit, welche einfach per Knopfdruck gestartet werden
können sollte.

Ein weiter Punkt welcher bei der User-Experience beachtet werden soll,
ist die zeitliche Konstante. Ein Spiel auf einem normalen Schachspiel
hat je nach Spielart kein Zeitlimit, dies kann für das gesamte Spiel
gelten oder auch für die Zeit zwischen einzelnen Zügen. Der autonome
Schachtisch soll es dem Spieler z.B. ermöglichen ein Spiel am Morgen zu
beginnen und dieses erst am nächsten Tag fortzusetzen.

\hypertarget{anforderungsanalyse}{%
\subsection{Anforderungsanalyse}\label{anforderungsanalyse}}

alle key requirements welcher der tisch haben soll

\hypertarget{machbarkeitsanalyse}{%
\subsection{Machbarkeitsanalyse}\label{machbarkeitsanalyse}}

welche technologien werden benötigt

\hypertarget{technologien-im-makerspace}{%
\subsubsection{Technologien im
Makerspace}\label{technologien-im-makerspace}}

stehen diese im makerspace zur verfüfung

\hypertarget{section}{%
\subsubsection{}\label{section}}

\hypertarget{erstellung-erster-prototyp}{%
\section{Erstellung erster Prototyp}\label{erstellung-erster-prototyp}}

\hypertarget{technologieauswahl-fuxfcr-ersten-protoypen}{%
\subsection{Technologieauswahl für ersten
Protoypen}\label{technologieauswahl-fuxfcr-ersten-protoypen}}

was stejt zur verfügung
