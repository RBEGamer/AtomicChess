Das Ziel der Arbeit ist es, einen autonomen Schachtisch zu entwickeln,
welcher in der Lage ist, Schachfiguren autonom zu bewegen und auf
Benutzerinteraktionen zu reagieren. Die Kernfrage der Arbeit bezieht
sich somit auf die Überprüfung der Ausführbarkeit inklusive Erstellung
und Umsetzung eines eingebetteten Systems und einer Cloud-Infrastruktur.
Der Schwerpunkt liegt dabei insbesondere auf der Programmierung des
eingebetteten Systems und dem Zusammenspiel von diesem mit einem aus dem
Internet erreichbaren Servers, welcher als Vermittlungsstelle zwischen
verschiedenen Schachtischen und anderen Endgeräten dient.

Zuerst werden die zum Zeitpunkt existierenden Ansätze und deren
Umsetzung beleuchtet. Hier wurde insbesondere darauf geachtet, die
Grenzen bestehender Systeme darzulegen und auf nur für dieses Projekt
zutreffende Funktionen zu vergleichen. Aus den Ergebnissen werden
anschließend die Anforderungen des autonomen Schachtischs abgeleitet,
welcher in dieser Arbeit umgesetzt werden sollen.

Anschließend wurden iterativ nacheinander zwei Prototypen umgesetzt,
welche sich vom mechanischen Design sowie des elektrischen stark
unterscheiden. Die entwickelte Software wurde hingegen so modular
entwickelt, dass diese auf beiden Prototypen zum Einsatz kommt. Die
unterschiedlichen Designs kommen dadurch zustande, dass in der ersten
Iteration des autonomen Schachtischs, nach einem Dauertest zahlreiche
verbesserungspotentiale erkannt wurden. Dies führte zu einer kompletten
Neugestaltung in der zweiten Iteration und somit wurde ein autonomer
Schachtisch entwickelt, welcher alle gestellten Anforderungen erfüllt.

Grundsätzlich ist festzuhalten, dass es sich beim Resultat der Arbeit um
kein finalisiertes Produkt, sondern um einen strukturellen Prototyp
handelt. Weitere Prüfungen, wie Nutzungsstatistiken oder
Sicherheitsprüfungen, müssten durchgeführt werden, ehe der Schachtisch
als kommerzielles Produkt betrachtet werden kann.

Das System und insbesondere der implementierte Cloud-Service sind online
erreichbar und erweiterbar. Dies ermöglicht unter anderem das Bauen
eines eigenen Tisches unter der Verwendung des AtomicChess Systems, aber
auch die Integration weiterer Komponenten. Erfahrene Entwickler können
somit das Spiel beliebig ausweiten oder sogar andere Spiele ergänzen.
Die für das Projekt entworfene Mechanik und Spielführung kann
dementsprechend auch für diverse andere Tischbrettspiele verwendet
werden.
