\hypertarget{abstract}{%
\section{Abstract}\label{abstract}}

Der gezielte Einsatz von Marketingkampagnen zur Vermarktung des eigenen
Produkts wird zunehmend verfolgt. Es wird wachsender Umsatz online
generiert, wobei insbesondere Social-Media-Kanäle bei der
Onlinevermarktung eine zentrale Rolle spielen. Daher ist es notwendig zu
verstehen, wie solche Marketingkampagnen konzipiert sind und wie sie
funktionieren.

Das Ziel der Forschung der vorliegenden Arbeit ist es zu beantworten,
welche Kriterien eine Marketingkampagne auf Social Media erfolgreich
machen. Dazu wird die folgende Forschungsfrage gestellt: ‚Wie kann eine
erfolgreiche Marketingkampagne für Onlinefotodruckunternehmen auf Social
Media geplant werden?`.

Um die Forschungsfrage zu beantworten, wurde eine quantitative Studie zu
aktuellen Druckgeschäftsanzeigen und deren Wirkung durchgeführt.
Spezifisch hat sich die Studie mit Anzeigen auf den Social-Media-Kanälen
Twitter, Facebook und Instagram beschäftigt. Es wurde untersucht, welche
Kriterien bei einer Anzeige erfüllt sein müssen, damit diese bei den
Nutzern erfolgreich ist. In der quantitativen Studie wurden den
Teilnehmenden geschlossene Fragen auf einer Skala von 1 bis 10 gestellt,
die im Anschluss ausgewertet wurden, wobei drei Altersklassen
berücksichtigt wurden: 15--29-Jährige, 30--45-Jährige und alle
Teilnehmenden ab 45 Jahren wurden je in eine Gruppe unterteilt.

Die Antworten auf die Fragebogen zeigen, dass die Altersgruppen von
30--45 Jahren und darüber im Durchschnitt am häufigsten auf die Anzeigen
von Onlinefotodruckunternehmen reagiert. Diese Anzeigen sind in erster
Linie auf Twitter und Facebook erfolgreich, weil diese Plattformen von
Personen in dieser Altersklasse am häufigsten genutzt werden. Jüngere
Menschen hingegen, die vorwiegend Instagram verwenden, reagieren
seltener auf die Anzeigen von Onlinefotodruckunternehmen. Eine
Social-Media-Kampagne bietet sich für Onlinefotodruckunternehmen also
insbesondere auf Twitter und Facebook mit der Fokussierung auf die
Altersgruppe ab 30 Jahren an.

Weiterführende Forschung im Bereich des Marketings für den
Onlinefotodruck könnte sich mit Anzeigenwerbung von Suchmaschinen
beschäftigen.
