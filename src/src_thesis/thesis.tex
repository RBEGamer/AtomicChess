

\documentclass [
% 	draft		% falls ohne Bilder gedruckt werden soll (Entwurf)
	]{scrbook}	% KOMA Klasse für Bücher

%%MOD%%
\usepackage{fhacmb}	% Style-File für Titelblatt. Ggf. bei Enning holen

\KOMAoptions{
	parskip=true,		% Absätze mit Abstand
	fontsize=12pt,		% Standardschriftgröße
	toc=flat,		% Inhaltsverzeichnis ohne Einzüge
	twoside=false,		% Einseitig setzen
	numbers=nodotatend,	% Nummerierungen nicht mit Punkt abschließen
% die folgenden Optionen nehmen die entsprechende Dinge ins Inhaltsverzeichnis auf
% mit der bei texlive vorhandenen aktuellen Version von pdflatex funkioniert es nicht mehr
% (bekannter Bug)
	toc=bibliography,	% Literaturverzeichnis ins Inhaltsverzeichnis
	toc=listof,		% Abbildung- und Tabellenverzeichnis ins Inhaltsverzeichnis
	toc=index,		% Stichwortverzeichnis ins Inhaltsverzeichnis
	}
%

\usepackage{amsmath,amssymb}






\IfFileExists{xcolor.sty}{
    \RequirePackage{xcolor}
}{
    \RequirePackage{color}
}


\providecommand{\tightlist}{%
  \setlength{\itemsep}{0pt}\setlength{\parskip}{0pt}}



\usepackage{footnotebackref}
\usepackage{graphicx}






%%
%% added
%%

%
% language specification
%
% If no language is specified, use English as the default main document language.
%
%%MOD%%



%
% for the background color of the title page
%

%
% break urls
%
\PassOptionsToPackage{hyphens}{url}

%
% When using babel or polyglossia with biblatex, loading csquotes is recommended
% to ensure that quoted texts are typeset according to the rules of your main language.
%
\usepackage{csquotes}


%
% variables for title, author and date
%
\usepackage{titling}





%
% ADD SYNTAX HIGHLIGHTING
%

%
%
% Listings
%
%
\usepackage{listings}
\newcommand{\passthrough}[1]{#1}
\lstset{defaultdialect=[5.3]Lua}
\lstset{defaultdialect=[x86masm]Assembler}
%
% general listing colors
%
\definecolor{listing-background}{HTML}{F7F7F7}
\definecolor{listing-rule}{HTML}{B3B2B3}
\definecolor{listing-numbers}{HTML}{B3B2B3}
\definecolor{listing-text-color}{HTML}{000000}
\definecolor{listing-keyword}{HTML}{435489}
\definecolor{listing-keyword-2}{HTML}{1284CA} % additional keywords
\definecolor{listing-keyword-3}{HTML}{9137CB} % additional keywords
\definecolor{listing-identifier}{HTML}{435489}
\definecolor{listing-string}{HTML}{00999A}
\definecolor{listing-comment}{HTML}{8E8E8E}

\lstdefinestyle{eisvogel_listing_style}{
  language         = java,
  numbers          = left,
  xleftmargin      = 2.7em,
  framexleftmargin = 2.5em,
  backgroundcolor  = \color{listing-background},
  basicstyle       = \color{listing-text-color}\linespread{1.0}\small\ttfamily{},
  breaklines       = true,
  frame            = single,
  framesep         = 0.19em,
  rulecolor        = \color{listing-rule},
  frameround       = ffff,
  tabsize          = 4,
  numberstyle      = \color{listing-numbers},
  aboveskip        = 1.0em,
  belowskip        = 0.1em,
  abovecaptionskip = 0em,
  belowcaptionskip = 1.0em,
  keywordstyle     = {\color{listing-keyword}\bfseries},
  keywordstyle     = {[2]\color{listing-keyword-2}\bfseries},
  keywordstyle     = {[3]\color{listing-keyword-3}\bfseries\itshape},
  sensitive        = true,
  identifierstyle  = \color{listing-identifier},
  commentstyle     = \color{listing-comment},
  stringstyle      = \color{listing-string},
  showstringspaces = false,
  escapeinside     = {/*@}{@*/}, % Allow LaTeX inside these special comments
  literate         =
  {á}{{\'a}}1 {é}{{\'e}}1 {í}{{\'i}}1 {ó}{{\'o}}1 {ú}{{\'u}}1
  {Á}{{\'A}}1 {É}{{\'E}}1 {Í}{{\'I}}1 {Ó}{{\'O}}1 {Ú}{{\'U}}1
  {à}{{\`a}}1 {è}{{\'e}}1 {ì}{{\`i}}1 {ò}{{\`o}}1 {ù}{{\`u}}1
  {À}{{\`A}}1 {È}{{\'E}}1 {Ì}{{\`I}}1 {Ò}{{\`O}}1 {Ù}{{\`U}}1
  {ä}{{\"a}}1 {ë}{{\"e}}1 {ï}{{\"i}}1 {ö}{{\"o}}1 {ü}{{\"u}}1
  {Ä}{{\"A}}1 {Ë}{{\"E}}1 {Ï}{{\"I}}1 {Ö}{{\"O}}1 {Ü}{{\"U}}1
  {â}{{\^a}}1 {ê}{{\^e}}1 {î}{{\^i}}1 {ô}{{\^o}}1 {û}{{\^u}}1
  {Â}{{\^A}}1 {Ê}{{\^E}}1 {Î}{{\^I}}1 {Ô}{{\^O}}1 {Û}{{\^U}}1
  {œ}{{\oe}}1 {Œ}{{\OE}}1 {æ}{{\ae}}1 {Æ}{{\AE}}1 {ß}{{\ss}}1
  {ç}{{\c c}}1 {Ç}{{\c C}}1 {ø}{{\o}}1 {å}{{\r a}}1 {Å}{{\r A}}1
  {€}{{\EUR}}1 {£}{{\pounds}}1 {«}{{\guillemotleft}}1
  {»}{{\guillemotright}}1 {ñ}{{\~n}}1 {Ñ}{{\~N}}1 {¿}{{?`}}1
  {…}{{\ldots}}1 {≥}{{>=}}1 {≤}{{<=}}1 {„}{{\glqq}}1 {“}{{\grqq}}1
  {”}{{''}}1
}
\lstset{style=eisvogel_listing_style}

%
% Java (Java SE 12, 2019-06-22)
%
\lstdefinelanguage{Java}{
  morekeywords={
    % normal keywords (without data types)
    abstract,assert,break,case,catch,class,continue,default,
    do,else,enum,exports,extends,final,finally,for,if,implements,
    import,instanceof,interface,module,native,new,package,private,
    protected,public,requires,return,static,strictfp,super,switch,
    synchronized,this,throw,throws,transient,try,volatile,while,
    % var is an identifier
    var
  },
  morekeywords={[2] % data types
    % primitive data types
    boolean,byte,char,double,float,int,long,short,
    % String
    String,
    % primitive wrapper types
    Boolean,Byte,Character,Double,Float,Integer,Long,Short
    % number types
    Number,AtomicInteger,AtomicLong,BigDecimal,BigInteger,DoubleAccumulator,DoubleAdder,LongAccumulator,LongAdder,Short,
    % other
    Object,Void,void
  },
  morekeywords={[3] % literals
    % reserved words for literal values
    null,true,false,
  },
  sensitive,
  morecomment  = [l]//,
  morecomment  = [s]{/*}{*/},
  morecomment  = [s]{/**}{*/},
  morestring   = [b]",
  morestring   = [b]',
}

\lstdefinelanguage{C++}{
  morekeywords={
    % normal keywords (without data types)
    abstract,assert,break,case,catch,class,continue,default,
    do,else,enum,exports,extends,final,finally,for,if,implements,
    import,instanceof,interface,module,native,new,package,private,
    protected,public,requires,return,static,strictfp,super,switch,
    synchronized,this,throw,throws,transient,try,volatile,while,
    const,this,template,struct,union,volatile,auto,inline,noexcept,not_eq,
    const_cast,extern,namespace,mutable,reflexpr,reinterpret_cast,
    static_cast,throw,typedef,typeid,wchar_t,xor_eq,or_eq,asm,std
    % var is an identifier
    var
  },
  morekeywords={[2] % data types
    % primitive data types
    bool,char,int,float,double,void,wchar_t,string,short,signed,long,unsigned
  },
  morekeywords={[3] % literals
    % reserved words for literal values
    null,true,false,
  },
  sensitive,
  morecomment  = [l]//,
  morecomment  = [s]{/*}{*/},
  morecomment  = [s]{/**}{*/},
  morestring   = [b]",
  morestring   = [b]',
}

\lstdefinelanguage{QML}{
  morekeywords={
    % normal keywords (without data types)
    default,required,readonly,property,function,
    import,qsTr,delegate
  },
  morekeywords={[2] % data types
    % primitive data types
    QtQuick,TextInput,Text,Connections,Rectangle,Item,Button,MenuManager,
    Image,BusyIndicator,GraphChart,ListView,AnimatedImage,Flickable,TextEdit,
    BorderImage,FocusScope,MouseArea,CheckBox,CheckDelegate,ComboBox,DelayButton,Dial,
    Frame,GroupBox,ItemDelegate,Label,Page,PageIndicator,Pane,ProgressBar,
    RadioButton,RadioDelegate,RangeSlider,RoundButton,ScrollView,Slider,SpinBox,StackView,
    SwipeDelegate,SwipeView,Switch,SwitchDelegate,TabBar,TabButton,TextArea,TextField,ToolBar,
    ToolButton,ToolSeperator,Tumbler,ColumnLayout,GridLayout,RowLayout,StackLayout,
    Column,Flow,Grid,Row,GridView,ListView,PathView,
    ColorAnimation,NumberAnimation,ParallelAnimation,PauseAnimation,PropertyAction,PropertyAnimation,
    ScriptAction,SequentialAnimation
  },
  morekeywords={[3] % literals
    % reserved words for literal values
    null,true,false,id,x,y,width,height,color,visible,objectName,target,
    horizontalAlignment,wrapMode,change,value´
  },
  sensitive,
  morecomment  = [l]//,
  morecomment  = [s]{/*}{*/},
  morecomment  = [s]{/**}{*/},
  morestring   = [b]",
  morestring   = [b]',
}


\lstdefinelanguage{JavaScript}{
  keywords={typeof, new, true, false, catch, function, return, null, catch, switch, var, if, in, while, do, else, case, break},
  %keywordstyle=\color{blue}\bfseries,
  ndkeywords={class, export, boolean, throw, implements, import, this},
  %ndkeywordstyle=\color{darkgray}\bfseries,
  %identifierstyle=\color{black},
  sensitive=false,
  comment=[l]{//},
  morecomment=[s]{/*}{*/},
  %commentstyle=\color{purple}\ttfamily,
  %stringstyle=\color{red}\ttfamily,
  morestring=[b]',
  morestring=[b]"
}

\usepackage{xcolor}
\colorlet{punct}{red!60!black}
\definecolor{background}{HTML}{EEEEEE}
\definecolor{delim}{RGB}{20,105,176}
\colorlet{numb}{magenta!60!black}
\lstdefinelanguage{JSON}{
    basicstyle=\normalfont\ttfamily,
    numbers=left,
    %numberstyle=\scriptsize,
    %stepnumber=1,  
    %showstringspaces=false,
    breaklines=true,
    literate=
     *{0}{{{\color{numb}0}}}{1}
      {1}{{{\color{numb}1}}}{1}
      {2}{{{\color{numb}2}}}{1}
      {3}{{{\color{numb}3}}}{1}
      {4}{{{\color{numb}4}}}{1}
      {5}{{{\color{numb}5}}}{1}
      {6}{{{\color{numb}6}}}{1}
      {7}{{{\color{numb}7}}}{1}
      {8}{{{\color{numb}8}}}{1}
      {9}{{{\color{numb}9}}}{1}
      {:}{{{\color{punct}{:}}}}{1}
      {,}{{{\color{punct}{,}}}}{1}
      {\{}{{{\color{delim}{\{}}}}{1}
      {\}}{{{\color{delim}{\}}}}}{1}
      {[}{{{\color{delim}{[}}}}{1}
      {]}{{{\color{delim}{]}}}}{1},
}



\lstdefinelanguage{XML}{
  morestring      = [b]",
  moredelim       = [s][\bfseries\color{listing-keyword}]{<}{\ },
  moredelim       = [s][\bfseries\color{listing-keyword}]{</}{>},
  moredelim       = [l][\bfseries\color{listing-keyword}]{/>},
  moredelim       = [l][\bfseries\color{listing-keyword}]{>},
  morecomment     = [s]{<?}{?>},
  morecomment     = [s]{<!--}{-->},
  commentstyle    = \color{listing-comment},
  stringstyle     = \color{listing-string},
  identifierstyle = \color{listing-identifier}
}




%%%%%% Immer benötigte Packages
%
\usepackage[T1]{fontenc}		% sonst funktioniert die Silbentrennung bei Umlauten nicht
\usepackage[utf8]{inputenc}	% Eingabedekodierung. Ermöglicht Umlaute. Achtung: Unbedingt mit Betreuer
				% Verwendung der Umlaute-Eingabemethode absprechen. Im Zweifel \"O für Ö
\usepackage[ngerman]{babel}	% Silbentrennung und Sprachanpassung
\usepackage{blindtext}		% Blindtext

%\usepackage[hidelinks]{hyperref}		% Sprungmarken, z.B. im Inhaltsverzeichnis auf Textpassagen



\usepackage{graphicx}		% Definiert o.a. \includegraphics
\usepackage[export]{adjustbox}
\let\oldincludegraphics\includegraphics
\renewcommand{\includegraphics}[2][]{%
  \oldincludegraphics[#1,max width=\linewidth]{#2}}



\usepackage{textcomp}		% Sonst funktioniert z.B. \texteuro nicht
\usepackage{scrlayer-scrpage}	% Package zum Definieren der Kopf- und Fußzeilen
\usepackage{amsmath,amssymb}		% Muss sein
\usepackage{mathrsfs} 	% Weitere Mathematik-Symbole, z.B. Laplace-L
%
%%%%% Anpassung an Formatvorlagen des Fachbereichs
%
\usepackage{helvet}		% Serifenlose Schrift ähnlich Arial
\renewcommand{\familydefault}{\sfdefault}	% als Standardschrift serifenlose Schrift verwenden
%
\usepackage{geometry} 		% Ränder direkt einstellen
\geometry{a4paper, top=20mm, left=30mm, right=20mm, bottom=25mm} % nach Vorgabe
\linespread{1.25} 		% Zeilenabstand nach Vorgabe
%
\usepackage{chngcntr}		% Ändert Verhalten von Countern
\counterwithout{figure}{section}	% Figure-Nummerierung nicht bei section-Wechsel zurücksetzen
\renewcommand{\thefigure}{\textbf\thechapter-\arabic{figure}}	% im Stil 3-2
%
%%%% für das Erzeugen von Grafiken mit Zeichenbefehlen
%
\usepackage{tikz}		% Grundpaket
\usetikzlibrary{shapes,arrows}	% einige Symbolpackages
\usepackage{tikz-cd}		% einige Symbolpackages

%
%%%% TABELLEN
%
\usepackage{colortbl}		% für die Hintergrundfarbe von Tabellen
\usepackage{paralist}		% Weitere Nummeriungsoptionen, z.b. alphabetisch für enumerate/itemize

\usepackage{longtable}% FOR PANDOC TABLE GENERATOR
\usepackage{booktabs} % FOR TOPRULE MIDRULE


%
%%%% UNORDERED LISTS
%
\renewcommand{\labelitemii}{$\circ$} % Bullets für itemize-Listen



%
%%%% PAGE AND FIGURE NUMBERING
%
\usepackage{chngcntr}		% Ändert Verhalten von Countern
\counterwithout{figure}{section}	% Figure-Nummerierung nicht bei section-Wechsel zurücksetzen
\renewcommand{\thefigure}{\textbf\thechapter-\arabic{figure}}	% im Stil 3-2

%\usepackage{subfig}		% Unterfigures mit eigenen Bildunterschriften

%\setcounter{tocdepth}{3} % SHOW SUB SUB SUB SUB SECTIONS


%
%%%% acronyms
%



\usepackage{pdflscape}




\usepackage[acronym]{glossaries}
\makeglossaries


\makeglossaries

\newacronym{html}{HTML}{Hypertext Markup Language}
\newacronym{js}{JS}{JavaScript}
\newacronym{css}{CSS}{Cascading Style Sheets}
\newacronym{ndefrtd}{NDEF-RTD}{NDEF Record Type Defintion}
\newacronym{http}{HTTP}{Hypertext Transfer Protocol}
\newacronym{https}{HTTPS}{Hypertext Transfer Protocol Secure}
\newacronym{ftp}{FTP}{File Transfer Protocol}
\newacronym{dns}{DNS}{Domain Name System}
\newacronym{aws}{AWS}{Amazon Web Services}
\newacronym{nfc}{NFC}{Near Field Communication}
\newacronym{rfid}{RFID}{Radio-Frequency Identification}
\newacronym{ci}{CI}{Continuous Integration}
\newacronym{cd}{CD}{Continuous Deployment}
\newacronym{iot}{iot}{Internet of Things}
\newacronym{mqtt}{mqtt}{Message Queuing Telemetry Transport}
\newacronym{rest}{REST}{Representational State Transfer}
\newacronym{api}{API}{Application Programming Interface}
\newacronym{cad}{CAD}{Computer-Aided Design}
\newacronym{rpi}{RPI}{Raspberry-Pi}
\newacronym{os}{OS}{Operating System}
\newacronym{vcs}{VCS}{Version Control System}
\newacronym{atc}{ATC}{Atomic Chess Table}
\newacronym{ui}{UI}{User Interface}
\newacronym{gui}{GUI}{Graphical User Interface}
\newacronym{ssh}{SSH}{Secure Shell}
\newacronym{ntp}{NTP}{Network Time Protocol}
\newacronym{sftp}{SFTP}{Secure File Transfer Protocol}
\newacronym{ide}{IDE}{Integrated Development Environment}
\newacronym{ipc}{IPC}{Interprocess Communication}
\newacronym{qml}{QML}{Qt Modeling Language}
\newacronym{dtb}{DTB}{Device Tree Binary}
\newacronym{dtc}{DTC}{Device Tree Compiler}
\newacronym{dtd}{DTD}{Device Tree Document}
\newacronym{ndef}{NDEF}{NFC Data Exchange Format}
\newacronym{gcc}{GCC}{GNU Compiler Collection}
\newacronym{gdb}{GDB}{GNU Debugger}
\newacronym{i2c}{I2C}{Inter-Integrated Circuit}
\newacronym{iic}{IIC}{Inter-Integrated Circuit}
\newacronym{spi}{SPI}{Serial Peripheral Interface}
\newacronym{usb}{USB}{Universal Serial Bus}
\newacronym{ufw}{ufw}{Uncomplicated Firewall}
\newacronym{pkg}{PKG}{Package}
\newacronym{ai}{AI}{Artificial Intelligence}
\newacronym{fdm}{FDM}{Fused Deposition Modeling}
\newacronym{sla}{SLA}{Stereolithography}
\newacronym{pla}{PLA}{Polylactic Acid}
\newacronym{abs}{ABS}{Acrylonitril-Butadien-Styrol-Copolymer}
\newacronym{tcp}{TCP}{Transmission Control Protocol}
\newacronym{udp}{UDP}{User Datagram Protocol}
\newacronym{stl}{STL}{Standard Triangle Language}
\newacronym{pcb}{PCB}{Printed Cirtcuit Board}
\newacronym{lan}{LAN}{Local Area Network}
\newacronym{wlan}{WLAN}{Wireless Local Area Network}
\newacronym{wan}{WAN}{Wild Area Network}
\newacronym{dc}{DC}{Direct Current}
\newacronym{ac}{AC}{Alternating Current}
\newacronym{iso}{ISO}{Internation Organisation for Standardization}
\newacronym{din}{din}{Deutsches Institut für Normung}
\newacronym{gpl}{GPL}{General Public License}
\newacronym{ble}{BLE}{Bluetooth Low Energy}
\newacronym{uci}{UCI}{Universal Chess Interface}
\newacronym{fen}{FEN}{Forsyth-Edwards-Notation}

%
%%%%%%%%%%  Angaben für Titelseite %%%%%%%%%%%%%%%%%%%%%%
%
% Angaben für Titelseite
\arbeitstyp{Bachelorarbeit}
\fachbereich{Elektrotechnik und Informationstechnik}
\studiengang{Informatik}
\vertiefung{}
\titel{Integration eines eingebetteten Systems in eine Cloud-Infrastruktur am Beispiel eines autonomen Spielfelds}
\autor{Marcel Werner Heinrich Friedrich Ochsendorf}
\matrnr{3120232}
\betreuer{Prof. Dr.-Ing. Thomas Dey}
\cobetreuer{Prof. Dr. Andreas Claßen}
\extbetreuer{}
\datum{11.06.2021}
\dank{}





%\hypersetup{draft} %DISBALE HYPERLINKS
% https://charly-lersteau.com/posts/2019-12-26-latex-hyperref-error-pdfendlink/
%\usepackage{etoolbox}

%\makeatletter
%\patchcmd\@combinedblfloats{\box\@outputbox}{\unvbox\@outputbox}{}{%
 %   \errmessage{\noexpand\@combinedblfloats could not be patched}%
%}%
%\makeatother

\usepackage{rotating}

\definecolor{LightCyan}{rgb}{0.88,1,1}
\definecolor{Gray}{gray}{0.9}
\usepackage[first=0,last=9]{lcg}
\newcommand{\ra}{\rand0.\arabic{rand}}


% ADD PAGE PACING FOR BETTER READABLE TABLES
\newcolumntype{g}{>{\columncolor{Gray}}c}
\renewcommand*{\arraystretch}{2} %2.0 IS  THE SPACING

\begin{document}



%%MOD%%
% Einige Anpassungen müssen nach \begin{document} stehen !!
\renewcaptionname{ngerman}{\figurename}{\textbf Bild} 	% Bild ... statt Abbildung ...
\renewcaptionname{ngerman}{\contentsname}{Inhalt}% Inhalt statt Inhaltsverzeichnis
%%%%%%%%%%%%%%%%%%%%%%%%%%%%%%%%%%%%%%%%%%%%%%%%%%%%%%%%%%%%
% Titel im FH Style
%%%%%%%%%%%%%%%%%%%%%%%%%%%%%%%%%%%%%%%%%%%%%%%%%%%%%%%%%%%%
\fhacmbtitle{\includegraphics[height=4cm]{fh_logo}}{5pt}{5pt}
%%%%%%%%%%%%%%%%%%%%%%%%%%%%%%%%%%%%%%%%%%%%%%%%%%%%%%%%%%%%
% Erklärung / Geheimhaltung
%%%%%%%%%%%%%%%%%%%%%%%%%%%%%%%%%%%%%%%%%%%%%%%%%%%%%%%%%%%%
%\frontmatter 	% Wenn der Hauptteil mit Seite 1 beginnen soll
\pagestyle{plain}








\frontmatter

%%%%%%%%%%%%%%%%%%%%%%%%%%%%%%%%%%%%%%%%%%%%%%%%%%%%%%%%%%%%
% Erklaerung
%%%%%%%%%%%%%%%%%%%%%%%%%%%%%%%%%%%%%%%%%%%%%%%%%%%%%%%%%%%%
\newpage

\hypertarget{erkluxe4rung}{%
\section*{Erklärung}\label{erkluxe4rung}}
Hiermit erkläre ich, dass ich die vorliegende Arbeit eigenständig und
ohne fremde Hilfe angefertigt habe. Textpassagen, die wörtlich oder dem
Sinn nach auf Publikationen oder Vorträgen anderer Autoren beruhen, sind
als solche kenntlich gemacht. Die Arbeit wurde bisher keiner anderen
Prüfungsbehörde vorgelegt und auch noch nicht veröffentlicht.

Aachen, den 14.06.2021 \_\_\_\_\_\_\_\_\_\_\_\_\_\_\_\_\_\_\_



%%%%%%%%%%%%%%%%%%%%%%%%%%%%%%%%%%%%%%%%%%%%%%%%%%%%%%%%%%%%
% ABSTRACT
%%%%%%%%%%%%%%%%%%%%%%%%%%%%%%%%%%%%%%%%%%%%%%%%%%%%%%%%%%%%
\newpage

\hypertarget{abstract}{%
\section*{Abstract}\label{abstract}}
\hypertarget{abstract}{%
\section{Abstract}\label{abstract}}

Die Kurzfassung gibt auf ein bis zwei Seiten die wesentlichen Inhalte
und Ergebnisse der Abschlussarbeit wieder.

Sie gliedert sich inhaltlich in

\begin{itemize}
\tightlist
\item
  das behandelte Gebiet,
\item
  das Zielder Arbeit,
\item
  die Untersuchungsmethode,
\item
  die Ergebnisse und
\item
  die Schlussfolgerungen.
\end{itemize}

Die Kurzfassung enthält keine Schlussfolgerungen oder Bewertungen, die
über die Inhal-te der Kapitel der Arbeit hinausgehen.

Alle Aussagen der Kurzfassung finden sich in aus-führlicher Form in der
Arbeit wieder. Die Kurzfassung erhält keine Kapitelnummer.






%%%%%%%%%%%%%%%%%%%%%%%%%%%%%%%%%%%%%%%%%%%%%%%%%%%%%%%%%%%%
% Inhaltsverzeichnis
%%%%%%%%%%%%%%%%%%%%%%%%%%%%%%%%%%%%%%%%%%%%%%%%%%%%%%%%%%%%
\tableofcontents


%%%%%%%%%%%%%%%%%%%%%%%%%%%%%%%%%%%%%%%%%%%%%%%%%%%%%%%%%%%%
% CONTENT
%%%%%%%%%%%%%%%%%%%%%%%%%%%%%%%%%%%%%%%%%%%%%%%%%%%%%%%%%%%%
\mainmatter	% Wenn der Hauptteil mit Seite 1 beginnen soll
\pagestyle{scrheadings}
%%%%%%%%%%%%%%% Anpassung des Seitenstils an FH-Layoutvorschrift %%%%%%%%%%%%
\renewcommand{\chaptermark}[1]{\markboth{\thechapter\hspace{1cm}#1}{}}	% Kapitel für Headerzeile neu definieren (ohne Nummer)
\chead{}		% Header Mitte löschen
\ihead{\leftmark}	% Kapitelbezeichnung links setzen
\renewcommand{\headfont}{\bfseries}	% Bold-Font für Headerzeile verwenden
\setheadsepline{0.5pt}





\hypertarget{vinaque-sanguine-metuenti-cuiquam-alcyone-fixus}{%
\section{Vinaque sanguine metuenti cuiquam Alcyone
fixus}\label{vinaque-sanguine-metuenti-cuiquam-alcyone-fixus}}

\hypertarget{aesculeae-domus-vincemur-et-veneris-adsuetus-lapsum}{%
\subsection{Aesculeae domus vincemur et Veneris adsuetus
lapsum}\label{aesculeae-domus-vincemur-et-veneris-adsuetus-lapsum}}

Lorem markdownum Letoia, et alios: figurae flectentem annis aliquid
Peneosque ab esse, obstat gravitate. Obscura atque coniuge, per de
coniunx, sibi \textbf{medias commentaque virgine} anima tamen comitemque
petis, sed. In Amphion vestros hamos ire arceor mandere spicula, in
licet aliquando.

\begin{lstlisting}[language=Java]
public class Example implements LoremIpsum {
    public static void main(String[] args) {
        if(args.length < 2) {
            System.out.println("Lorem ipsum dolor sit amet");
        }
    } // Obscura atque coniuge, per de coniunx
}
\end{lstlisting}

Listing: TEST

\begin{lstlisting}[language={C++}]
// Your First C++ Program

#include <iostream>

int main() {
    std::cout << "Hello World!";
    return 0;
}

}
\end{lstlisting}

Porrigitur et Pallas nuper longusque cratere habuisse sepulcro pectore
fertur. Laudat ille auditi; vertitur iura tum nepotis causa; motus. Diva
virtus! Acrota destruitis vos iubet quo et classis excessere Scyrumve
spiro subitusque mente Pirithoi abstulit, lapides.

\hypertarget{lydia-caelo-recenti-haerebat-lacerum-ratae-at}{%
\subsection{Lydia caelo recenti haerebat lacerum ratae
at}\label{lydia-caelo-recenti-haerebat-lacerum-ratae-at}}

Te concepit pollice fugit vias alumno \textbf{oras} quam potest
\href{http://example.com\#rursus}{rursus} optat. Non evadere orbem
equorum, spatiis, vel pede inter si.

\begin{enumerate}
\def\labelenumi{\arabic{enumi}.}
\tightlist
\item
  De neque iura aquis
\item
  Frangitur gaudia mihi eo umor terrae quos
\item
  Recens diffudit ille tantum
\end{enumerate}

\begin{equation}\label{eq:neighbor-propability}
    p_{ij}(t) = \frac{\ell_j(t) - \ell_i(t)}{\sum_{k \in N_i(t)}^{} \ell_k(t) - \ell_i(t)}
\end{equation}

Tamen condeturque saxa Pallorque num et ferarum promittis inveni lilia
iuvencae adessent arbor. Florente perque at condeturque saxa et ferarum
promittis tendebat. Armos nisi obortas refugit me.

\begin{quote}
Et nepotes poterat, se qui. Euntem ego pater desuetaque aethera
Maeandri, et \href{http://example.com\#Dardanio_geminaque}{Dardanio
geminaque} cernit. Lassaque poenas nec, manifesta \(\pi r^2\) mirantia
captivarum prohibebant scelerato gradus unusque dura.
\end{quote}

\begin{itemize}
\tightlist
\item
  Permulcens flebile simul
\item
  Iura tum nepotis causa motus diva virtus Acrota. Tamen condeturque
  saxa Pallorque num et ferarum promittis inveni lilia iuvencae adessent
  arbor. Florente perque at ire arcum.
\end{itemize}

\hypertarget{latex-table-with-caption}{%
\subsection{LaTeX Table with Caption}\label{latex-table-with-caption}}

At vero eos et accusam et justo duo dolores et ea rebum. Stet clita kasd
gubergren, no sea takimata sanctus est Lorem ipsum dolor sit amet. Lorem
ipsum dolor sit amet, consetetur sadipscing elitr.

\begin{longtable}[]{@{}lll@{}}
\caption{Verschiedene Bewegungsalgorithmen im Vergleich}\tabularnewline
\toprule
\begin{minipage}[b]{0.29\columnwidth}\raggedright
ALGORITHM\_V1\_TRAVEL\_TIME {[}s{]}\strut
\end{minipage} & \begin{minipage}[b]{0.29\columnwidth}\raggedright
ALGORITHM\_V2\_TRAVEL\_TIME {[}s{]}\strut
\end{minipage} & \begin{minipage}[b]{0.34\columnwidth}\raggedright
TRAVEL\_DISTANCE {[}FIELDS\_DIAGONAL{]}\strut
\end{minipage}\tabularnewline
\midrule
\endfirsthead
\toprule
\begin{minipage}[b]{0.29\columnwidth}\raggedright
ALGORITHM\_V1\_TRAVEL\_TIME {[}s{]}\strut
\end{minipage} & \begin{minipage}[b]{0.29\columnwidth}\raggedright
ALGORITHM\_V2\_TRAVEL\_TIME {[}s{]}\strut
\end{minipage} & \begin{minipage}[b]{0.34\columnwidth}\raggedright
TRAVEL\_DISTANCE {[}FIELDS\_DIAGONAL{]}\strut
\end{minipage}\tabularnewline
\midrule
\endhead
\begin{minipage}[t]{0.29\columnwidth}\raggedright
7.20\strut
\end{minipage} & \begin{minipage}[t]{0.29\columnwidth}\raggedright
2.56\strut
\end{minipage} & \begin{minipage}[t]{0.34\columnwidth}\raggedright
1\strut
\end{minipage}\tabularnewline
\begin{minipage}[t]{0.29\columnwidth}\raggedright
11.56\strut
\end{minipage} & \begin{minipage}[t]{0.29\columnwidth}\raggedright
6,20\strut
\end{minipage} & \begin{minipage}[t]{0.34\columnwidth}\raggedright
3\strut
\end{minipage}\tabularnewline
\begin{minipage}[t]{0.29\columnwidth}\raggedright
12,27\strut
\end{minipage} & \begin{minipage}[t]{0.29\columnwidth}\raggedright
7,06\strut
\end{minipage} & \begin{minipage}[t]{0.34\columnwidth}\raggedright
5\strut
\end{minipage}\tabularnewline
\begin{minipage}[t]{0.29\columnwidth}\raggedright
14,39\strut
\end{minipage} & \begin{minipage}[t]{0.29\columnwidth}\raggedright
6,56\strut
\end{minipage} & \begin{minipage}[t]{0.34\columnwidth}\raggedright
8\strut
\end{minipage}\tabularnewline
\bottomrule
\end{longtable}

\hypertarget{image-with-caption}{%
\subsection{Image with Caption}\label{image-with-caption}}

\begin{figure}
\centering
\includegraphics{images/ATC_Calibration_Guide.png}
\caption{Kalibrierungeschema der Mechanik zeigt welche Abstände in der
Konfiguration eigetragen werden müssen}
\end{figure}








%%%%%%%%%%%%%%%%%%%%%%%%%%%%%%%%%%%%%%%%%%%%%%%%%%%%%%%%%%%%
% REFERENZEN
%%%%%%%%%%%%%%%%%%%%%%%%%%%%%%%%%%%%%%%%%%%%%%%%%%%%%%%%%%%%
%% Verschiedene Versionen, nach DIN 1505 zu zitieren
\bibliographystyle{plaindin}
%\interlinepenalty=10000
\bibliography{thesis_references}


%%%%%%%%%%%%%%%%%%%%%%%%%%%%%%%%%%%%%%%%%%%%%%%%%%%%%%%%%%%%
% ACRONYM VERZEICHNIS
%%%%%%%%%%%%%%%%%%%%%%%%%%%%%%%%%%%%%%%%%%%%%%%%%%%%%%%%%%%%
\printglossaries

%%%%%%%%%%%%%%%%%%%%%%%%%%%%%%%%%%%%%%%%%%%%%%%%%%%%%%%%%%%%
% ABBILDUNGSVERZEICHNIS
%%%%%%%%%%%%%%%%%%%%%%%%%%%%%%%%%%%%%%%%%%%%%%%%%%%%%%%%%%%%
\listoffigures

%%%%%%%%%%%%%%%%%%%%%%%%%%%%%%%%%%%%%%%%%%%%%%%%%%%%%%%%%%%%
% ABBILDUNGSVERZEICHNIS
%%%%%%%%%%%%%%%%%%%%%%%%%%%%%%%%%%%%%%%%%%%%%%%%%%%%%%%%%%%%
\listoftables

%%%%%%%%%%%%%%%%%%%%%%%%%%%%%%%%%%%%%%%%%%%%%%%%%%%%%%%%%%%%
% ANHANG
%%%%%%%%%%%%%%%%%%%%%%%%%%%%%%%%%%%%%%%%%%%%%%%%%%%%%%%%%%%%
%\newpage
%\appendix % ANHANG EINLEITEN
%\hypertarget{anhang}{%
\section{Anhang}\label{anhang}}



\end{document}
