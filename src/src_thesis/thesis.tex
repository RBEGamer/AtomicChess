
\documentclass[
  paper=a4,
  ,captions=tableheading
]{scrbook}

%%MOD%%
\usepackage{fhacmb}	% Style-File für Titelblatt. Ggf. bei Enning holen

\KOMAoptions{
	parskip=true,		% Absätze mit Abstand
	fontsize=12pt,		% Standardschriftgröße 
	toc=flat,		% Inhaltsverzeichnis ohne Einzüge
	twoside=false,		% Einseitig setzen
	numbers=nodotatend,	% Nummerierungen nicht mit Punkt abschließen
% die folgenden Optionen nehmen die entsprechende Dinge ins Inhaltsverzeichnis auf
% mit der bei texlive vorhandenen aktuellen Version von pdflatex funkioniert es nicht mehr
% (bekannter Bug)
%	toc=bibliography,	% Literaturverzeichnis ins Inhaltsverzeichnis
%	toc=listof,		% Abbildung- und Tabellenverzeichnis ins Inhaltsverzeichnis
%	toc=index,		% Stichwortverzeichnis ins Inhaltsverzeichnis
	}
%

\usepackage{amsmath,amssymb}
\usepackage{lmodern}
\usepackage{setspace}
\setstretch{1.2}
\usepackage{ifxetex,ifluatex}
\ifnum 0\ifxetex 1\fi\ifluatex 1\fi=0 % if pdftex
  \usepackage[T1]{fontenc}
  \usepackage[utf8]{inputenc}
  \usepackage{textcomp} % provide euro and other symbols
\else % if luatex or xetex
  \usepackage{unicode-math}
  \defaultfontfeatures{Scale=MatchLowercase}
  \defaultfontfeatures[\rmfamily]{Ligatures=TeX,Scale=1}
\fi
% Use upquote if available, for straight quotes in verbatim environments
\IfFileExists{upquote.sty}{\usepackage{upquote}}{}
\IfFileExists{microtype.sty}{% use microtype if available
  \usepackage[]{microtype}
  \UseMicrotypeSet[protrusion]{basicmath} % disable protrusion for tt fonts
}{}
\makeatletter
\@ifundefined{KOMAClassName}{% if non-KOMA class
  \IfFileExists{parskip.sty}{%
    \usepackage{parskip}
  }{% else
    \setlength{\parindent}{0pt}
    \setlength{\parskip}{6pt plus 2pt minus 1pt}}
}{% if KOMA class
  \KOMAoptions{parskip=half}}
\makeatother
\usepackage{xcolor}
\definecolor{default-linkcolor}{HTML}{A50000}
\definecolor{default-filecolor}{HTML}{A50000}
\definecolor{default-citecolor}{HTML}{4077C0}
\definecolor{default-urlcolor}{HTML}{4077C0}
\IfFileExists{xurl.sty}{\usepackage{xurl}}{} % add URL line breaks if available
%%MOD%%
%\IfFileExists{bookmark.sty}{\usepackage{bookmark}}{\usepackage{hyperref}}
%\hypersetup{
%  hidelinks,
%  breaklinks=true,
%  pdfcreator={}} %%MOD%%
\urlstyle{same} % disable monospaced font for URLs


%%MOD%%
%\usepackage[margin=2.5cm,includehead=true,includefoot=true,centering,]{geometry}



\usepackage{listings}
\newcommand{\passthrough}[1]{#1}
\lstset{defaultdialect=[5.3]Lua}
\lstset{defaultdialect=[x86masm]Assembler}
% add backlinks to footnote references, cf. https://tex.stackexchange.com/questions/302266/make-footnote-clickable-both-ways
\usepackage{footnotebackref}
\usepackage{graphicx}
\makeatletter
\def\maxwidth{\ifdim\Gin@nat@width>\linewidth\linewidth\else\Gin@nat@width\fi}
\def\maxheight{\ifdim\Gin@nat@height>\textheight\textheight\else\Gin@nat@height\fi}
\makeatother
% Scale images if necessary, so that they will not overflow the page
% margins by default, and it is still possible to overwrite the defaults
% using explicit options in \includegraphics[width, height, ...]{}
\setkeys{Gin}{width=\maxwidth,height=\maxheight,keepaspectratio}
% Set default figure placement to htbp
\makeatletter
\def\fps@figure{htbp}
\makeatother
\setlength{\emergencystretch}{3em} % prevent overfull lines
\providecommand{\tightlist}{%
  \setlength{\itemsep}{0pt}\setlength{\parskip}{0pt}}
\setcounter{secnumdepth}{-\maxdimen} % remove section numbering

% Make use of float-package and set default placement for figures to H.
% The option H means 'PUT IT HERE' (as  opposed to the standard h option which means 'You may put it here if you like').
\usepackage{float}
\floatplacement{figure}{H}

\ifluatex
  \usepackage{selnolig}  % disable illegal ligatures
\fi

\author{}
\date{}



%%
%% added
%%

%
% language specification
%
% If no language is specified, use English as the default main document language.
%
%%MOD%%



%
% for the background color of the title page
%

%
% break urls
%
\PassOptionsToPackage{hyphens}{url}

%
% When using babel or polyglossia with biblatex, loading csquotes is recommended
% to ensure that quoted texts are typeset according to the rules of your main language.
%
\usepackage{csquotes}

%
% captions
%
\definecolor{caption-color}{HTML}{777777}
\usepackage[font={stretch=1.2}, textfont={color=caption-color}, position=top, skip=4mm, labelfont=bf, singlelinecheck=false, justification=raggedright]{caption}
\setcapindent{0em}

%
% blockquote
%
\definecolor{blockquote-border}{RGB}{221,221,221}
\definecolor{blockquote-text}{RGB}{119,119,119}
\usepackage{mdframed}
\newmdenv[rightline=false,bottomline=false,topline=false,linewidth=3pt,linecolor=blockquote-border,skipabove=\parskip]{customblockquote}
\renewenvironment{quote}{\begin{customblockquote}\list{}{\rightmargin=0em\leftmargin=0em}%
\item\relax\color{blockquote-text}\ignorespaces}{\unskip\unskip\endlist\end{customblockquote}}

%
% Source Sans Pro as the de­fault font fam­ily
% Source Code Pro for monospace text
%
% 'default' option sets the default
% font family to Source Sans Pro, not \sfdefault.
%
\ifnum 0\ifxetex 1\fi\ifluatex 1\fi=0 % if pdftex
    \usepackage[default]{sourcesanspro}
  \usepackage{sourcecodepro}
  \else % if not pdftex
    \usepackage[default]{sourcesanspro}
  \usepackage{sourcecodepro}

  % XeLaTeX specific adjustments for straight quotes: https://tex.stackexchange.com/a/354887
  % This issue is already fixed (see https://github.com/silkeh/latex-sourcecodepro/pull/5) but the
  % fix is still unreleased.
  % TODO: Remove this workaround when the new version of sourcecodepro is released on CTAN.
  \ifxetex
    \makeatletter
    \defaultfontfeatures[\ttfamily]
      { Numbers   = \sourcecodepro@figurestyle,
        Scale     = \SourceCodePro@scale,
        Extension = .otf }
    \setmonofont
      [ UprightFont    = *-\sourcecodepro@regstyle,
        ItalicFont     = *-\sourcecodepro@regstyle It,
        BoldFont       = *-\sourcecodepro@boldstyle,
        BoldItalicFont = *-\sourcecodepro@boldstyle It ]
      {SourceCodePro}
    \makeatother
  \fi
  \fi

%
% heading color
%
\definecolor{heading-color}{RGB}{40,40,40}
\addtokomafont{section}{\color{heading-color}}
% When using the classes report, scrreprt, book,
% scrbook or memoir, uncomment the following line.
%\addtokomafont{chapter}{\color{heading-color}}

%
% variables for title, author and date
%
\usepackage{titling}
\title{}
\author{}
\date{}

%
% tables
%

%
% remove paragraph indention
%
\setlength{\parindent}{0pt}
\setlength{\parskip}{6pt plus 2pt minus 1pt}
\setlength{\emergencystretch}{3em}  % prevent overfull lines

%
%
% Listings
%
%


%
% general listing colors
%
\definecolor{listing-background}{HTML}{F7F7F7}
\definecolor{listing-rule}{HTML}{B3B2B3}
\definecolor{listing-numbers}{HTML}{B3B2B3}
\definecolor{listing-text-color}{HTML}{000000}
\definecolor{listing-keyword}{HTML}{435489}
\definecolor{listing-keyword-2}{HTML}{1284CA} % additional keywords
\definecolor{listing-keyword-3}{HTML}{9137CB} % additional keywords
\definecolor{listing-identifier}{HTML}{435489}
\definecolor{listing-string}{HTML}{00999A}
\definecolor{listing-comment}{HTML}{8E8E8E}

\lstdefinestyle{eisvogel_listing_style}{
  language         = java,
  numbers          = left,
  xleftmargin      = 2.7em,
  framexleftmargin = 2.5em,
  backgroundcolor  = \color{listing-background},
  basicstyle       = \color{listing-text-color}\linespread{1.0}\small\ttfamily{},
  breaklines       = true,
  frame            = single,
  framesep         = 0.19em,
  rulecolor        = \color{listing-rule},
  frameround       = ffff,
  tabsize          = 4,
  numberstyle      = \color{listing-numbers},
  aboveskip        = 1.0em,
  belowskip        = 0.1em,
  abovecaptionskip = 0em,
  belowcaptionskip = 1.0em,
  keywordstyle     = {\color{listing-keyword}\bfseries},
  keywordstyle     = {[2]\color{listing-keyword-2}\bfseries},
  keywordstyle     = {[3]\color{listing-keyword-3}\bfseries\itshape},
  sensitive        = true,
  identifierstyle  = \color{listing-identifier},
  commentstyle     = \color{listing-comment},
  stringstyle      = \color{listing-string},
  showstringspaces = false,
  escapeinside     = {/*@}{@*/}, % Allow LaTeX inside these special comments
  literate         =
  {á}{{\'a}}1 {é}{{\'e}}1 {í}{{\'i}}1 {ó}{{\'o}}1 {ú}{{\'u}}1
  {Á}{{\'A}}1 {É}{{\'E}}1 {Í}{{\'I}}1 {Ó}{{\'O}}1 {Ú}{{\'U}}1
  {à}{{\`a}}1 {è}{{\'e}}1 {ì}{{\`i}}1 {ò}{{\`o}}1 {ù}{{\`u}}1
  {À}{{\`A}}1 {È}{{\'E}}1 {Ì}{{\`I}}1 {Ò}{{\`O}}1 {Ù}{{\`U}}1
  {ä}{{\"a}}1 {ë}{{\"e}}1 {ï}{{\"i}}1 {ö}{{\"o}}1 {ü}{{\"u}}1
  {Ä}{{\"A}}1 {Ë}{{\"E}}1 {Ï}{{\"I}}1 {Ö}{{\"O}}1 {Ü}{{\"U}}1
  {â}{{\^a}}1 {ê}{{\^e}}1 {î}{{\^i}}1 {ô}{{\^o}}1 {û}{{\^u}}1
  {Â}{{\^A}}1 {Ê}{{\^E}}1 {Î}{{\^I}}1 {Ô}{{\^O}}1 {Û}{{\^U}}1
  {œ}{{\oe}}1 {Œ}{{\OE}}1 {æ}{{\ae}}1 {Æ}{{\AE}}1 {ß}{{\ss}}1
  {ç}{{\c c}}1 {Ç}{{\c C}}1 {ø}{{\o}}1 {å}{{\r a}}1 {Å}{{\r A}}1
  {€}{{\EUR}}1 {£}{{\pounds}}1 {«}{{\guillemotleft}}1
  {»}{{\guillemotright}}1 {ñ}{{\~n}}1 {Ñ}{{\~N}}1 {¿}{{?`}}1
  {…}{{\ldots}}1 {≥}{{>=}}1 {≤}{{<=}}1 {„}{{\glqq}}1 {“}{{\grqq}}1
  {”}{{''}}1
}
\lstset{style=eisvogel_listing_style}

%
% Java (Java SE 12, 2019-06-22)
%
\lstdefinelanguage{Java}{
  morekeywords={
    % normal keywords (without data types)
    abstract,assert,break,case,catch,class,continue,default,
    do,else,enum,exports,extends,final,finally,for,if,implements,
    import,instanceof,interface,module,native,new,package,private,
    protected,public,requires,return,static,strictfp,super,switch,
    synchronized,this,throw,throws,transient,try,volatile,while,
    % var is an identifier
    var
  },
  morekeywords={[2] % data types
    % primitive data types
    boolean,byte,char,double,float,int,long,short,
    % String
    String,
    % primitive wrapper types
    Boolean,Byte,Character,Double,Float,Integer,Long,Short
    % number types
    Number,AtomicInteger,AtomicLong,BigDecimal,BigInteger,DoubleAccumulator,DoubleAdder,LongAccumulator,LongAdder,Short,
    % other
    Object,Void,void
  },
  morekeywords={[3] % literals
    % reserved words for literal values
    null,true,false,
  },
  sensitive,
  morecomment  = [l]//,
  morecomment  = [s]{/*}{*/},
  morecomment  = [s]{/**}{*/},
  morestring   = [b]",
  morestring   = [b]',
}

\lstdefinelanguage{C++}{
  morekeywords={
    % normal keywords (without data types)
    abstract,assert,break,case,catch,class,continue,default,
    do,else,enum,exports,extends,final,finally,for,if,implements,
    import,instanceof,interface,module,native,new,package,private,
    protected,public,requires,return,static,strictfp,super,switch,
    synchronized,this,throw,throws,transient,try,volatile,while,
    const,this,template,struct,union,volatile,auto,inline,noexcept,not_eq,
    const_cast,extern,namespace,mutable,reflexpr,reinterpret_cast,
    static_cast,throw,typedef,typeid,wchar_t,xor_eq,or_eq,asm,std
    % var is an identifier
    var
  },
  morekeywords={[2] % data types
    % primitive data types
    bool,char,int,float,double,void,wchar_t,string,short,signed,long,unsigned
  },
  morekeywords={[3] % literals
    % reserved words for literal values
    null,true,false,
  },
  sensitive,
  morecomment  = [l]//,
  morecomment  = [s]{/*}{*/},
  morecomment  = [s]{/**}{*/},
  morestring   = [b]",
  morestring   = [b]',
}

\lstdefinelanguage{QML}{
  morekeywords={
    % normal keywords (without data types)
    default,required,readonly,property,function,
    import,qsTr,delegate
  },
  morekeywords={[2] % data types
    % primitive data types
    QtQuick,TextInput,Text,Connections,Rectangle,Item,Button,MenuManager,
    Image,BusyIndicator,GraphChart,ListView,AnimatedImage,Flickable,TextEdit,
    BorderImage,FocusScope,MouseArea,CheckBox,CheckDelegate,ComboBox,DelayButton,Dial,
    Frame,GroupBox,ItemDelegate,Label,Page,PageIndicator,Pane,ProgressBar,
    RadioButton,RadioDelegate,RangeSlider,RoundButton,ScrollView,Slider,SpinBox,StackView,
    SwipeDelegate,SwipeView,Switch,SwitchDelegate,TabBar,TabButton,TextArea,TextField,ToolBar,
    ToolButton,ToolSeperator,Tumbler,ColumnLayout,GridLayout,RowLayout,StackLayout,
    Column,Flow,Grid,Row,GridView,ListView,PathView,
    ColorAnimation,NumberAnimation,ParallelAnimation,PauseAnimation,PropertyAction,PropertyAnimation,
    ScriptAction,SequentialAnimation
  },
  morekeywords={[3] % literals
    % reserved words for literal values
    null,true,false,id,x,y,width,height,color,visible,objectName,target,
    horizontalAlignment,wrapMode,change,value´
  },
  sensitive,
  morecomment  = [l]//,
  morecomment  = [s]{/*}{*/},
  morecomment  = [s]{/**}{*/},
  morestring   = [b]",
  morestring   = [b]',
}


\lstdefinelanguage{JavaScript}{
  keywords={typeof, new, true, false, catch, function, return, null, catch, switch, var, if, in, while, do, else, case, break},
  %keywordstyle=\color{blue}\bfseries,
  ndkeywords={class, export, boolean, throw, implements, import, this},
  %ndkeywordstyle=\color{darkgray}\bfseries,
  %identifierstyle=\color{black},
  sensitive=false,
  comment=[l]{//},
  morecomment=[s]{/*}{*/},
  %commentstyle=\color{purple}\ttfamily,
  %stringstyle=\color{red}\ttfamily,
  morestring=[b]',
  morestring=[b]"
}

\usepackage{xcolor}
\colorlet{punct}{red!60!black}
\definecolor{background}{HTML}{EEEEEE}
\definecolor{delim}{RGB}{20,105,176}
\colorlet{numb}{magenta!60!black}
\lstdefinelanguage{JSON}{
    basicstyle=\normalfont\ttfamily,
    numbers=left,
    %numberstyle=\scriptsize,
    %stepnumber=1,  
    %showstringspaces=false,
    breaklines=true,
    literate=
     *{0}{{{\color{numb}0}}}{1}
      {1}{{{\color{numb}1}}}{1}
      {2}{{{\color{numb}2}}}{1}
      {3}{{{\color{numb}3}}}{1}
      {4}{{{\color{numb}4}}}{1}
      {5}{{{\color{numb}5}}}{1}
      {6}{{{\color{numb}6}}}{1}
      {7}{{{\color{numb}7}}}{1}
      {8}{{{\color{numb}8}}}{1}
      {9}{{{\color{numb}9}}}{1}
      {:}{{{\color{punct}{:}}}}{1}
      {,}{{{\color{punct}{,}}}}{1}
      {\{}{{{\color{delim}{\{}}}}{1}
      {\}}{{{\color{delim}{\}}}}}{1}
      {[}{{{\color{delim}{[}}}}{1}
      {]}{{{\color{delim}{]}}}}{1},
}



\lstdefinelanguage{XML}{
  morestring      = [b]",
  moredelim       = [s][\bfseries\color{listing-keyword}]{<}{\ },
  moredelim       = [s][\bfseries\color{listing-keyword}]{</}{>},
  moredelim       = [l][\bfseries\color{listing-keyword}]{/>},
  moredelim       = [l][\bfseries\color{listing-keyword}]{>},
  morecomment     = [s]{<?}{?>},
  morecomment     = [s]{<!--}{-->},
  commentstyle    = \color{listing-comment},
  stringstyle     = \color{listing-string},
  identifierstyle = \color{listing-identifier}
}

%
% header and footer
%
\usepackage[headsepline,footsepline]{scrlayer-scrpage}

\newpairofpagestyles{eisvogel-header-footer}{
  \clearpairofpagestyles
  \ihead[]{}
  \chead[]{}
  \ohead[]{}
  \ifoot[\thepage]{}
  \cfoot[]{}
  \ofoot[]{\thepage}
  \addtokomafont{pageheadfoot}{\upshape}
}
\pagestyle{eisvogel-header-footer}

%%
%% end added
%%

%%%%%% Immer benötigte Packages
%
\usepackage[T1]{fontenc}		% sonst funktioniert die Silbentrennung bei Umlauten nicht
\usepackage[utf8]{inputenc}	% Eingabedekodierung. Ermöglicht Umlaute. Achtung: Unbedingt mit Betreuer
				% Verwendung der Umlaute-Eingabemethode absprechen. Im Zweifel \"O für Ö
\usepackage[ngerman]{babel}	% Silbentrennung und Sprachanpassung
\usepackage{blindtext}		% Blindtext

%\usepackage[hidelinks]{hyperref}		% Sprungmarken, z.B. im Inhaltsverzeichnis auf Textpassagen



\usepackage{graphicx}		% Definiert o.a. \includegraphics
\usepackage[export]{adjustbox}
\let\oldincludegraphics\includegraphics
\renewcommand{\includegraphics}[2][]{%
  \oldincludegraphics[#1,max width=\linewidth]{#2}}



\usepackage{textcomp}		% Sonst funktioniert z.B. \texteuro nicht
\usepackage{scrlayer-scrpage}	% Package zum Definieren der Kopf- und Fußzeilen
\usepackage{amsmath}		% Muss sein
\usepackage{mathrsfs} 	% Weitere Mathematik-Symbole, z.B. Laplace-L
%
%%%%% Anpassung an Formatvorlagen des Fachbereichs
%
\usepackage{helvet}		% Serifenlose Schrift ähnlich Arial
\renewcommand{\familydefault}{\sfdefault}	% als Standardschrift serifenlose Schrift verwenden
%
\usepackage{geometry} 		% Ränder direkt einstellen
\geometry{a4paper, top=20mm, left=30mm, right=20mm, bottom=25mm} % nach Vorgabe
\linespread{1.25} 		% Zeilenabstand nach Vorgabe
%
\usepackage{chngcntr}		% Ändert Verhalten von Countern
\counterwithout{figure}{section}	% Figure-Nummerierung nicht bei section-Wechsel zurücksetzen
\renewcommand{\thefigure}{\textbf\thechapter-\arabic{figure}}	% im Stil 3-2
%
%%%% für das Erzeugen von Grafiken mit Zeichenbefehlen
%
\usepackage{tikz}		% Grundpaket
\usetikzlibrary{shapes,arrows}	% einige Symbolpackages
\usepackage{tikz-cd}		% einige Symbolpackages

\usepackage{longtable}% FOR PANDOC TABLE GENERATOR
\usepackage{booktabs} % FOR TOPRULE MIDRULE




%\setkeys{Gin}%{width=\ScaleWidthIfNeeded,height=\ScaleHeightIfNeeded,keepaspectratio}%

%%MOD%%

\usepackage{helvet}		% Serifenlose Schrift ähnlich Arial
\renewcommand{\familydefault}{\sfdefault}	% als Standardschrift serifenlose Schrift verwenden


\usepackage{chngcntr}		% Ändert Verhalten von Countern
\counterwithout{figure}{section}	% Figure-Nummerierung nicht bei section-Wechsel zurücksetzen
\renewcommand{\thefigure}{\textbf\thechapter-\arabic{figure}}	% im Stil 3-2

\usepackage{colortbl}		% für die Hintergrundfarbe von Tabellen
\usepackage{paralist}		% Weitere Nummeriungsoptionen, z.b. alphabetisch für enumerate/itemize
\usepackage{verbatim}		% Verbesserte verbatim-Umgebung (z.b. Programm-Listings)
%\usepackage{subfig}		% Unterfigures mit eigenen Bildunterschriften




%
%%%%%%%%%%  Angaben für Titelseite %%%%%%%%%%%%%%%%%%%%%%
%
% Angaben für Titelseite
\arbeitstyp{Bachelorarbeit}
\fachbereich{Elektrotechnik und Informationstechnik}
\studiengang{Informatik}
\vertiefung{}
\titel{Integration eines eingebetteten Systems in eine Cloud-Infrastruktur am Beispiel eines autonomen Spielfelds}
\autor{Marcel Werner Heinrich Friedrich Ochsendorf}
\matrnr{3120232}
\betreuer{Prof. Dr.-Ing. Thomas Dey}
\cobetreuer{Prof. Dr. Andreas Claßen}
\extbetreuer{}
\datum{11.06.2021}
\dank{}


\begin{document}

%%MOD%%
% Einige Anpassungen müssen nach \begin{document} stehen !!
\renewcaptionname{ngerman}{\figurename}{\textbf Bild} 	% Bild ... statt Abbildung ... 
\renewcaptionname{ngerman}{\contentsname}{Inhalt}% Inhalt statt Inhaltsverzeichnis
%%%%%%%%%%%%%%%%%%%%%%%%%%%%%%%%%%%%%%%%%%%%%%%%%%%%%%%%%%%%
% Titel im FH Style
%%%%%%%%%%%%%%%%%%%%%%%%%%%%%%%%%%%%%%%%%%%%%%%%%%%%%%%%%%%%
\fhacmbtitle{\includegraphics[height=4cm]{fh_logo}}{5pt}{5pt}
%%%%%%%%%%%%%%%%%%%%%%%%%%%%%%%%%%%%%%%%%%%%%%%%%%%%%%%%%%%%
% Erklärung / Geheimhaltung
%%%%%%%%%%%%%%%%%%%%%%%%%%%%%%%%%%%%%%%%%%%%%%%%%%%%%%%%%%%%
\frontmatter 	% Wenn der Hauptteil mit Seite 1 beginnen soll
\pagestyle{plain}


%%%%%%%%%%%%%%%%%%%%%%%%%%%%%%%%%%%%%%%%%%%%%%%%%%%%%%%%%%%%
% ABSTRACT
%%%%%%%%%%%%%%%%%%%%%%%%%%%%%%%%%%%%%%%%%%%%%%%%%%%%%%%%%%%%
\newpage
\hypertarget{abstract}{%
\section{Abstract}\label{abstract}}

Der gezielte Einsatz von Marketingkampagnen zur Vermarktung des eigenen
Produkts wird zunehmend verfolgt. Es wird wachsender Umsatz online
generiert, wobei insbesondere Social-Media-Kanäle bei der
Onlinevermarktung eine zentrale Rolle spielen. Daher ist es notwendig zu
verstehen, wie solche Marketingkampagnen konzipiert sind und wie sie
funktionieren.

Das Ziel der Forschung der vorliegenden Arbeit ist es zu beantworten,
welche Kriterien eine Marketingkampagne auf Social Media erfolgreich
machen. Dazu wird die folgende Forschungsfrage gestellt: ‚Wie kann eine
erfolgreiche Marketingkampagne für Onlinefotodruckunternehmen auf Social
Media geplant werden?`.

Um die Forschungsfrage zu beantworten, wurde eine quantitative Studie zu
aktuellen Druckgeschäftsanzeigen und deren Wirkung durchgeführt.
Spezifisch hat sich die Studie mit Anzeigen auf den Social-Media-Kanälen
Twitter, Facebook und Instagram beschäftigt. Es wurde untersucht, welche
Kriterien bei einer Anzeige erfüllt sein müssen, damit diese bei den
Nutzern erfolgreich ist. In der quantitativen Studie wurden den
Teilnehmenden geschlossene Fragen auf einer Skala von 1 bis 10 gestellt,
die im Anschluss ausgewertet wurden, wobei drei Altersklassen
berücksichtigt wurden: 15--29-Jährige, 30--45-Jährige und alle
Teilnehmenden ab 45 Jahren wurden je in eine Gruppe unterteilt.

Die Antworten auf die Fragebogen zeigen, dass die Altersgruppen von
30--45 Jahren und darüber im Durchschnitt am häufigsten auf die Anzeigen
von Onlinefotodruckunternehmen reagiert. Diese Anzeigen sind in erster
Linie auf Twitter und Facebook erfolgreich, weil diese Plattformen von
Personen in dieser Altersklasse am häufigsten genutzt werden. Jüngere
Menschen hingegen, die vorwiegend Instagram verwenden, reagieren
seltener auf die Anzeigen von Onlinefotodruckunternehmen. Eine
Social-Media-Kampagne bietet sich für Onlinefotodruckunternehmen also
insbesondere auf Twitter und Facebook mit der Fokussierung auf die
Altersgruppe ab 30 Jahren an.

Weiterführende Forschung im Bereich des Marketings für den
Onlinefotodruck könnte sich mit Anzeigenwerbung von Suchmaschinen
beschäftigen.

\newpage

%%%%%%%%%%%%%%%%%%%%%%%%%%%%%%%%%%%%%%%%%%%%%%%%%%%%%%%%%%%%
% Erklaerung
%%%%%%%%%%%%%%%%%%%%%%%%%%%%%%%%%%%%%%%%%%%%%%%%%%%%%%%%%%%%
\newpage
\input{declaration}
\newpage

%%%%%%%%%%%%%%%%%%%%%%%%%%%%%%%%%%%%%%%%%%%%%%%%%%%%%%%%%%%%
% Inhaltsverzeichnis
%%%%%%%%%%%%%%%%%%%%%%%%%%%%%%%%%%%%%%%%%%%%%%%%%%%%%%%%%%%%
\newpage
\tableofcontents
\newpage

%%%%%%%%%%%%%%%%%%%%%%%%%%%%%%%%%%%%%%%%%%%%%%%%%%%%%%%%%%%%
% CONTENT
%%%%%%%%%%%%%%%%%%%%%%%%%%%%%%%%%%%%%%%%%%%%%%%%%%%%%%%%%%%%
\mainmatter	% Wenn der Hauptteil mit Seite 1 beginnen soll
\pagestyle{scrheadings}
%%%%%%%%%%%%%%% Anpassung des Seitenstils an FH-Layoutvorschrift %%%%%%%%%%%%
\renewcommand{\chaptermark}[1]{\markboth{\thechapter\hspace{1cm}#1}{}}	% Kapitel für Headerzeile neu definieren (ohne Nummer)
\chead{}		% Header Mitte löschen
\ihead{\leftmark}	% Kapitelbezeichnung links setzen 
\renewcommand{\headfont}{\bfseries}	% Bold-Font für Headerzeile verwenden
\setheadsepline{0.5pt}





\hypertarget{vinaque-sanguine-metuenti-cuiquam-alcyone-fixus}{%
\section{Vinaque sanguine metuenti cuiquam Alcyone
fixus}\label{vinaque-sanguine-metuenti-cuiquam-alcyone-fixus}}

\hypertarget{aesculeae-domus-vincemur-et-veneris-adsuetus-lapsum}{%
\subsection{Aesculeae domus vincemur et Veneris adsuetus
lapsum}\label{aesculeae-domus-vincemur-et-veneris-adsuetus-lapsum}}

Lorem markdownum Letoia, et alios: figurae flectentem annis aliquid
Peneosque ab esse, obstat gravitate. Obscura atque coniuge, per de
coniunx, sibi \textbf{medias commentaque virgine} anima tamen comitemque
petis, sed. In Amphion vestros hamos ire arceor mandere spicula, in
licet aliquando.

\begin{lstlisting}[language=Java]
public class Example implements LoremIpsum {
    public static void main(String[] args) {
        if(args.length < 2) {
            System.out.println("Lorem ipsum dolor sit amet");
        }
    } // Obscura atque coniuge, per de coniunx
}
\end{lstlisting}

Listing: TEST

\begin{lstlisting}[language={C++}]
// Your First C++ Program

#include <iostream>

int main() {
    std::cout << "Hello World!";
    return 0;
}

}
\end{lstlisting}

Porrigitur et Pallas nuper longusque cratere habuisse sepulcro pectore
fertur. Laudat ille auditi; vertitur iura tum nepotis causa; motus. Diva
virtus! Acrota destruitis vos iubet quo et classis excessere Scyrumve
spiro subitusque mente Pirithoi abstulit, lapides.

\hypertarget{lydia-caelo-recenti-haerebat-lacerum-ratae-at}{%
\subsection{Lydia caelo recenti haerebat lacerum ratae
at}\label{lydia-caelo-recenti-haerebat-lacerum-ratae-at}}

Te concepit pollice fugit vias alumno \textbf{oras} quam potest
\href{http://example.com\#rursus}{rursus} optat. Non evadere orbem
equorum, spatiis, vel pede inter si.

\begin{enumerate}
\def\labelenumi{\arabic{enumi}.}
\tightlist
\item
  De neque iura aquis
\item
  Frangitur gaudia mihi eo umor terrae quos
\item
  Recens diffudit ille tantum
\end{enumerate}

\begin{equation}\label{eq:neighbor-propability}
    p_{ij}(t) = \frac{\ell_j(t) - \ell_i(t)}{\sum_{k \in N_i(t)}^{} \ell_k(t) - \ell_i(t)}
\end{equation}

Tamen condeturque saxa Pallorque num et ferarum promittis inveni lilia
iuvencae adessent arbor. Florente perque at condeturque saxa et ferarum
promittis tendebat. Armos nisi obortas refugit me.

\begin{quote}
Et nepotes poterat, se qui. Euntem ego pater desuetaque aethera
Maeandri, et \href{http://example.com\#Dardanio_geminaque}{Dardanio
geminaque} cernit. Lassaque poenas nec, manifesta \(\pi r^2\) mirantia
captivarum prohibebant scelerato gradus unusque dura.
\end{quote}

\begin{itemize}
\tightlist
\item
  Permulcens flebile simul
\item
  Iura tum nepotis causa motus diva virtus Acrota. Tamen condeturque
  saxa Pallorque num et ferarum promittis inveni lilia iuvencae adessent
  arbor. Florente perque at ire arcum.
\end{itemize}

\hypertarget{latex-table-with-caption}{%
\subsection{LaTeX Table with Caption}\label{latex-table-with-caption}}

At vero eos et accusam et justo duo dolores et ea rebum. Stet clita kasd
gubergren, no sea takimata sanctus est Lorem ipsum dolor sit amet. Lorem
ipsum dolor sit amet, consetetur sadipscing elitr.

\begin{longtable}[]{@{}lll@{}}
\caption{Verschiedene Bewegungsalgorithmen im Vergleich}\tabularnewline
\toprule
\begin{minipage}[b]{0.29\columnwidth}\raggedright
ALGORITHM\_V1\_TRAVEL\_TIME {[}s{]}\strut
\end{minipage} & \begin{minipage}[b]{0.29\columnwidth}\raggedright
ALGORITHM\_V2\_TRAVEL\_TIME {[}s{]}\strut
\end{minipage} & \begin{minipage}[b]{0.34\columnwidth}\raggedright
TRAVEL\_DISTANCE {[}FIELDS\_DIAGONAL{]}\strut
\end{minipage}\tabularnewline
\midrule
\endfirsthead
\toprule
\begin{minipage}[b]{0.29\columnwidth}\raggedright
ALGORITHM\_V1\_TRAVEL\_TIME {[}s{]}\strut
\end{minipage} & \begin{minipage}[b]{0.29\columnwidth}\raggedright
ALGORITHM\_V2\_TRAVEL\_TIME {[}s{]}\strut
\end{minipage} & \begin{minipage}[b]{0.34\columnwidth}\raggedright
TRAVEL\_DISTANCE {[}FIELDS\_DIAGONAL{]}\strut
\end{minipage}\tabularnewline
\midrule
\endhead
\begin{minipage}[t]{0.29\columnwidth}\raggedright
7.20\strut
\end{minipage} & \begin{minipage}[t]{0.29\columnwidth}\raggedright
2.56\strut
\end{minipage} & \begin{minipage}[t]{0.34\columnwidth}\raggedright
1\strut
\end{minipage}\tabularnewline
\begin{minipage}[t]{0.29\columnwidth}\raggedright
11.56\strut
\end{minipage} & \begin{minipage}[t]{0.29\columnwidth}\raggedright
6,20\strut
\end{minipage} & \begin{minipage}[t]{0.34\columnwidth}\raggedright
3\strut
\end{minipage}\tabularnewline
\begin{minipage}[t]{0.29\columnwidth}\raggedright
12,27\strut
\end{minipage} & \begin{minipage}[t]{0.29\columnwidth}\raggedright
7,06\strut
\end{minipage} & \begin{minipage}[t]{0.34\columnwidth}\raggedright
5\strut
\end{minipage}\tabularnewline
\begin{minipage}[t]{0.29\columnwidth}\raggedright
14,39\strut
\end{minipage} & \begin{minipage}[t]{0.29\columnwidth}\raggedright
6,56\strut
\end{minipage} & \begin{minipage}[t]{0.34\columnwidth}\raggedright
8\strut
\end{minipage}\tabularnewline
\bottomrule
\end{longtable}

\hypertarget{image-with-caption}{%
\subsection{Image with Caption}\label{image-with-caption}}

\begin{figure}
\centering
\includegraphics{images/ATC_Calibration_Guide.png}
\caption{Kalibrierungeschema der Mechanik zeigt welche Abstände in der
Konfiguration eigetragen werden müssen}
\end{figure}


Wenn Sie auf Quellen, wie z.B. diese Webreferenz \cite{Bohner2001} oder auch dieses Buch \cite{Murrenhoff} verweisen, erscheinen die entsprechenden Titel perfekt formatiert im Literaturverzeichnis.

Damit alle Referenzen stimmen, muss man ggf. pdflatex und bibtex mehrfach aufrufen. Sicher ist die folgende Reihenfolge
\begin{enumerate}
\item pdflatex. Erzeugt die *.aux-Datei, die bibtex benötigt
\item bibtex. Erzeugt Dateien, die pdflatex einbindet
\item pdflatex
\item pdflatex (Ggf. auf Fehlermeldungen schauen. Nach dem zweiten Lauf sollte alles okay sein)
\end{enumerate}








\medskip
%% Verschiedene Versionen, nach DIN 1505 zu zitieren
\bibliographystyle{plaindin}
\bibliography{thesis_references}


\listoffigures


\listoftables

\newpage
\appendix
\input{attachments}


\end{document}
