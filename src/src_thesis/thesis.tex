
\documentclass[
  paper=a4,
  ,captions=tableheading
]{scrartcl}

%%MOD%%
\usepackage{fhacmb}	% Style-File für Titelblatt. Ggf. bei Enning holen



\usepackage{amsmath,amssymb}
\usepackage{lmodern}
\usepackage{setspace}
\setstretch{1.2}
\usepackage{ifxetex,ifluatex}
\ifnum 0\ifxetex 1\fi\ifluatex 1\fi=0 % if pdftex
  \usepackage[T1]{fontenc}
  \usepackage[utf8]{inputenc}
  \usepackage{textcomp} % provide euro and other symbols
\else % if luatex or xetex
  \usepackage{unicode-math}
  \defaultfontfeatures{Scale=MatchLowercase}
  \defaultfontfeatures[\rmfamily]{Ligatures=TeX,Scale=1}
\fi
% Use upquote if available, for straight quotes in verbatim environments
\IfFileExists{upquote.sty}{\usepackage{upquote}}{}
\IfFileExists{microtype.sty}{% use microtype if available
  \usepackage[]{microtype}
  \UseMicrotypeSet[protrusion]{basicmath} % disable protrusion for tt fonts
}{}
\makeatletter
\@ifundefined{KOMAClassName}{% if non-KOMA class
  \IfFileExists{parskip.sty}{%
    \usepackage{parskip}
  }{% else
    \setlength{\parindent}{0pt}
    \setlength{\parskip}{6pt plus 2pt minus 1pt}}
}{% if KOMA class
  \KOMAoptions{parskip=half}}
\makeatother
\usepackage{xcolor}
\definecolor{default-linkcolor}{HTML}{A50000}
\definecolor{default-filecolor}{HTML}{A50000}
\definecolor{default-citecolor}{HTML}{4077C0}
\definecolor{default-urlcolor}{HTML}{4077C0}
\IfFileExists{xurl.sty}{\usepackage{xurl}}{} % add URL line breaks if available
%%MOD%%
%\IfFileExists{bookmark.sty}{\usepackage{bookmark}}{\usepackage{hyperref}}
%\hypersetup{
%  hidelinks,
%  breaklinks=true,
%  pdfcreator={}} %%MOD%%
\urlstyle{same} % disable monospaced font for URLs


%%MOD%%
%\usepackage[margin=2.5cm,includehead=true,includefoot=true,centering,]{geometry}



\usepackage{listings}
\newcommand{\passthrough}[1]{#1}
\lstset{defaultdialect=[5.3]Lua}
\lstset{defaultdialect=[x86masm]Assembler}
% add backlinks to footnote references, cf. https://tex.stackexchange.com/questions/302266/make-footnote-clickable-both-ways
\usepackage{footnotebackref}
\usepackage{graphicx}
\makeatletter
\def\maxwidth{\ifdim\Gin@nat@width>\linewidth\linewidth\else\Gin@nat@width\fi}
\def\maxheight{\ifdim\Gin@nat@height>\textheight\textheight\else\Gin@nat@height\fi}
\makeatother
% Scale images if necessary, so that they will not overflow the page
% margins by default, and it is still possible to overwrite the defaults
% using explicit options in \includegraphics[width, height, ...]{}
\setkeys{Gin}{width=\maxwidth,height=\maxheight,keepaspectratio}
% Set default figure placement to htbp
\makeatletter
\def\fps@figure{htbp}
\makeatother
\setlength{\emergencystretch}{3em} % prevent overfull lines
\providecommand{\tightlist}{%
  \setlength{\itemsep}{0pt}\setlength{\parskip}{0pt}}
\setcounter{secnumdepth}{-\maxdimen} % remove section numbering

% Make use of float-package and set default placement for figures to H.
% The option H means 'PUT IT HERE' (as  opposed to the standard h option which means 'You may put it here if you like').
\usepackage{float}
\floatplacement{figure}{H}

\ifluatex
  \usepackage{selnolig}  % disable illegal ligatures
\fi

\author{}
\date{}



%%
%% added
%%

%
% language specification
%
% If no language is specified, use English as the default main document language.
%
%%MOD%%



%
% for the background color of the title page
%

%
% break urls
%
\PassOptionsToPackage{hyphens}{url}

%
% When using babel or polyglossia with biblatex, loading csquotes is recommended
% to ensure that quoted texts are typeset according to the rules of your main language.
%
\usepackage{csquotes}

%
% captions
%
\definecolor{caption-color}{HTML}{777777}
\usepackage[font={stretch=1.2}, textfont={color=caption-color}, position=top, skip=4mm, labelfont=bf, singlelinecheck=false, justification=raggedright]{caption}
\setcapindent{0em}

%
% blockquote
%
\definecolor{blockquote-border}{RGB}{221,221,221}
\definecolor{blockquote-text}{RGB}{119,119,119}
\usepackage{mdframed}
\newmdenv[rightline=false,bottomline=false,topline=false,linewidth=3pt,linecolor=blockquote-border,skipabove=\parskip]{customblockquote}
\renewenvironment{quote}{\begin{customblockquote}\list{}{\rightmargin=0em\leftmargin=0em}%
\item\relax\color{blockquote-text}\ignorespaces}{\unskip\unskip\endlist\end{customblockquote}}

%
% Source Sans Pro as the de­fault font fam­ily
% Source Code Pro for monospace text
%
% 'default' option sets the default
% font family to Source Sans Pro, not \sfdefault.
%
\ifnum 0\ifxetex 1\fi\ifluatex 1\fi=0 % if pdftex
    \usepackage[default]{sourcesanspro}
  \usepackage{sourcecodepro}
  \else % if not pdftex
    \usepackage[default]{sourcesanspro}
  \usepackage{sourcecodepro}

  % XeLaTeX specific adjustments for straight quotes: https://tex.stackexchange.com/a/354887
  % This issue is already fixed (see https://github.com/silkeh/latex-sourcecodepro/pull/5) but the
  % fix is still unreleased.
  % TODO: Remove this workaround when the new version of sourcecodepro is released on CTAN.
  \ifxetex
    \makeatletter
    \defaultfontfeatures[\ttfamily]
      { Numbers   = \sourcecodepro@figurestyle,
        Scale     = \SourceCodePro@scale,
        Extension = .otf }
    \setmonofont
      [ UprightFont    = *-\sourcecodepro@regstyle,
        ItalicFont     = *-\sourcecodepro@regstyle It,
        BoldFont       = *-\sourcecodepro@boldstyle,
        BoldItalicFont = *-\sourcecodepro@boldstyle It ]
      {SourceCodePro}
    \makeatother
  \fi
  \fi

%
% heading color
%
\definecolor{heading-color}{RGB}{40,40,40}
\addtokomafont{section}{\color{heading-color}}
% When using the classes report, scrreprt, book,
% scrbook or memoir, uncomment the following line.
%\addtokomafont{chapter}{\color{heading-color}}

%
% variables for title, author and date
%
\usepackage{titling}
\title{}
\author{}
\date{}

%
% tables
%

%
% remove paragraph indention
%
\setlength{\parindent}{0pt}
\setlength{\parskip}{6pt plus 2pt minus 1pt}
\setlength{\emergencystretch}{3em}  % prevent overfull lines

%
%
% Listings
%
%


%
% general listing colors
%
\definecolor{listing-background}{HTML}{F7F7F7}
\definecolor{listing-rule}{HTML}{B3B2B3}
\definecolor{listing-numbers}{HTML}{B3B2B3}
\definecolor{listing-text-color}{HTML}{000000}
\definecolor{listing-keyword}{HTML}{435489}
\definecolor{listing-keyword-2}{HTML}{1284CA} % additional keywords
\definecolor{listing-keyword-3}{HTML}{9137CB} % additional keywords
\definecolor{listing-identifier}{HTML}{435489}
\definecolor{listing-string}{HTML}{00999A}
\definecolor{listing-comment}{HTML}{8E8E8E}

\lstdefinestyle{eisvogel_listing_style}{
  language         = java,
  numbers          = left,
  xleftmargin      = 2.7em,
  framexleftmargin = 2.5em,
  backgroundcolor  = \color{listing-background},
  basicstyle       = \color{listing-text-color}\linespread{1.0}\small\ttfamily{},
  breaklines       = true,
  frame            = single,
  framesep         = 0.19em,
  rulecolor        = \color{listing-rule},
  frameround       = ffff,
  tabsize          = 4,
  numberstyle      = \color{listing-numbers},
  aboveskip        = 1.0em,
  belowskip        = 0.1em,
  abovecaptionskip = 0em,
  belowcaptionskip = 1.0em,
  keywordstyle     = {\color{listing-keyword}\bfseries},
  keywordstyle     = {[2]\color{listing-keyword-2}\bfseries},
  keywordstyle     = {[3]\color{listing-keyword-3}\bfseries\itshape},
  sensitive        = true,
  identifierstyle  = \color{listing-identifier},
  commentstyle     = \color{listing-comment},
  stringstyle      = \color{listing-string},
  showstringspaces = false,
  escapeinside     = {/*@}{@*/}, % Allow LaTeX inside these special comments
  literate         =
  {á}{{\'a}}1 {é}{{\'e}}1 {í}{{\'i}}1 {ó}{{\'o}}1 {ú}{{\'u}}1
  {Á}{{\'A}}1 {É}{{\'E}}1 {Í}{{\'I}}1 {Ó}{{\'O}}1 {Ú}{{\'U}}1
  {à}{{\`a}}1 {è}{{\'e}}1 {ì}{{\`i}}1 {ò}{{\`o}}1 {ù}{{\`u}}1
  {À}{{\`A}}1 {È}{{\'E}}1 {Ì}{{\`I}}1 {Ò}{{\`O}}1 {Ù}{{\`U}}1
  {ä}{{\"a}}1 {ë}{{\"e}}1 {ï}{{\"i}}1 {ö}{{\"o}}1 {ü}{{\"u}}1
  {Ä}{{\"A}}1 {Ë}{{\"E}}1 {Ï}{{\"I}}1 {Ö}{{\"O}}1 {Ü}{{\"U}}1
  {â}{{\^a}}1 {ê}{{\^e}}1 {î}{{\^i}}1 {ô}{{\^o}}1 {û}{{\^u}}1
  {Â}{{\^A}}1 {Ê}{{\^E}}1 {Î}{{\^I}}1 {Ô}{{\^O}}1 {Û}{{\^U}}1
  {œ}{{\oe}}1 {Œ}{{\OE}}1 {æ}{{\ae}}1 {Æ}{{\AE}}1 {ß}{{\ss}}1
  {ç}{{\c c}}1 {Ç}{{\c C}}1 {ø}{{\o}}1 {å}{{\r a}}1 {Å}{{\r A}}1
  {€}{{\EUR}}1 {£}{{\pounds}}1 {«}{{\guillemotleft}}1
  {»}{{\guillemotright}}1 {ñ}{{\~n}}1 {Ñ}{{\~N}}1 {¿}{{?`}}1
  {…}{{\ldots}}1 {≥}{{>=}}1 {≤}{{<=}}1 {„}{{\glqq}}1 {“}{{\grqq}}1
  {”}{{''}}1
}
\lstset{style=eisvogel_listing_style}

%
% Java (Java SE 12, 2019-06-22)
%
\lstdefinelanguage{Java}{
  morekeywords={
    % normal keywords (without data types)
    abstract,assert,break,case,catch,class,continue,default,
    do,else,enum,exports,extends,final,finally,for,if,implements,
    import,instanceof,interface,module,native,new,package,private,
    protected,public,requires,return,static,strictfp,super,switch,
    synchronized,this,throw,throws,transient,try,volatile,while,
    % var is an identifier
    var
  },
  morekeywords={[2] % data types
    % primitive data types
    boolean,byte,char,double,float,int,long,short,
    % String
    String,
    % primitive wrapper types
    Boolean,Byte,Character,Double,Float,Integer,Long,Short
    % number types
    Number,AtomicInteger,AtomicLong,BigDecimal,BigInteger,DoubleAccumulator,DoubleAdder,LongAccumulator,LongAdder,Short,
    % other
    Object,Void,void
  },
  morekeywords={[3] % literals
    % reserved words for literal values
    null,true,false,
  },
  sensitive,
  morecomment  = [l]//,
  morecomment  = [s]{/*}{*/},
  morecomment  = [s]{/**}{*/},
  morestring   = [b]",
  morestring   = [b]',
}

\lstdefinelanguage{C++}{
  morekeywords={
    % normal keywords (without data types)
    abstract,assert,break,case,catch,class,continue,default,
    do,else,enum,exports,extends,final,finally,for,if,implements,
    import,instanceof,interface,module,native,new,package,private,
    protected,public,requires,return,static,strictfp,super,switch,
    synchronized,this,throw,throws,transient,try,volatile,while,
    const,this,template,struct,union,volatile,auto,inline,noexcept,not_eq,
    const_cast,extern,namespace,mutable,reflexpr,reinterpret_cast,
    static_cast,throw,typedef,typeid,wchar_t,xor_eq,or_eq,asm,std
    % var is an identifier
    var
  },
  morekeywords={[2] % data types
    % primitive data types
    bool,char,int,float,double,void,wchar_t,string,short,signed,long,unsigned
  },
  morekeywords={[3] % literals
    % reserved words for literal values
    null,true,false,
  },
  sensitive,
  morecomment  = [l]//,
  morecomment  = [s]{/*}{*/},
  morecomment  = [s]{/**}{*/},
  morestring   = [b]",
  morestring   = [b]',
}

\lstdefinelanguage{QML}{
  morekeywords={
    % normal keywords (without data types)
    default,required,readonly,property,function,
    import,qsTr,delegate
  },
  morekeywords={[2] % data types
    % primitive data types
    QtQuick,TextInput,Text,Connections,Rectangle,Item,Button,MenuManager,
    Image,BusyIndicator,GraphChart,ListView,AnimatedImage,Flickable,TextEdit,
    BorderImage,FocusScope,MouseArea,CheckBox,CheckDelegate,ComboBox,DelayButton,Dial,
    Frame,GroupBox,ItemDelegate,Label,Page,PageIndicator,Pane,ProgressBar,
    RadioButton,RadioDelegate,RangeSlider,RoundButton,ScrollView,Slider,SpinBox,StackView,
    SwipeDelegate,SwipeView,Switch,SwitchDelegate,TabBar,TabButton,TextArea,TextField,ToolBar,
    ToolButton,ToolSeperator,Tumbler,ColumnLayout,GridLayout,RowLayout,StackLayout,
    Column,Flow,Grid,Row,GridView,ListView,PathView,
    ColorAnimation,NumberAnimation,ParallelAnimation,PauseAnimation,PropertyAction,PropertyAnimation,
    ScriptAction,SequentialAnimation
  },
  morekeywords={[3] % literals
    % reserved words for literal values
    null,true,false,id,x,y,width,height,color,visible,objectName,target,
    horizontalAlignment,wrapMode,change,value´
  },
  sensitive,
  morecomment  = [l]//,
  morecomment  = [s]{/*}{*/},
  morecomment  = [s]{/**}{*/},
  morestring   = [b]",
  morestring   = [b]',
}


\lstdefinelanguage{JavaScript}{
  keywords={typeof, new, true, false, catch, function, return, null, catch, switch, var, if, in, while, do, else, case, break},
  %keywordstyle=\color{blue}\bfseries,
  ndkeywords={class, export, boolean, throw, implements, import, this},
  %ndkeywordstyle=\color{darkgray}\bfseries,
  identifierstyle=\color{black},
  sensitive=false,
  comment=[l]{//},
  morecomment=[s]{/*}{*/},
  %commentstyle=\color{purple}\ttfamily,
  %stringstyle=\color{red}\ttfamily,
  morestring=[b]',
  morestring=[b]"
}

\usepackage{xcolor}
\colorlet{punct}{red!60!black}
\definecolor{background}{HTML}{EEEEEE}
\definecolor{delim}{RGB}{20,105,176}
\colorlet{numb}{magenta!60!black}
\lstdefinelanguage{JSON}{
    basicstyle=\normalfont\ttfamily,
    numbers=left,
    %numberstyle=\scriptsize,
    %stepnumber=1,  
    %showstringspaces=false,
    breaklines=true,
    literate=
     *{0}{{{\color{numb}0}}}{1}
      {1}{{{\color{numb}1}}}{1}
      {2}{{{\color{numb}2}}}{1}
      {3}{{{\color{numb}3}}}{1}
      {4}{{{\color{numb}4}}}{1}
      {5}{{{\color{numb}5}}}{1}
      {6}{{{\color{numb}6}}}{1}
      {7}{{{\color{numb}7}}}{1}
      {8}{{{\color{numb}8}}}{1}
      {9}{{{\color{numb}9}}}{1}
      {:}{{{\color{punct}{:}}}}{1}
      {,}{{{\color{punct}{,}}}}{1}
      {\{}{{{\color{delim}{\{}}}}{1}
      {\}}{{{\color{delim}{\}}}}}{1}
      {[}{{{\color{delim}{[}}}}{1}
      {]}{{{\color{delim}{]}}}}{1},
}



\lstdefinelanguage{XML}{
  morestring      = [b]",
  moredelim       = [s][\bfseries\color{listing-keyword}]{<}{\ },
  moredelim       = [s][\bfseries\color{listing-keyword}]{</}{>},
  moredelim       = [l][\bfseries\color{listing-keyword}]{/>},
  moredelim       = [l][\bfseries\color{listing-keyword}]{>},
  morecomment     = [s]{<?}{?>},
  morecomment     = [s]{<!--}{-->},
  commentstyle    = \color{listing-comment},
  stringstyle     = \color{listing-string},
  identifierstyle = \color{listing-identifier}
}

%
% header and footer
%
\usepackage[headsepline,footsepline]{scrlayer-scrpage}

\newpairofpagestyles{eisvogel-header-footer}{
  \clearpairofpagestyles
  \ihead[]{}
  \chead[]{}
  \ohead[]{}
  \ifoot[\thepage]{}
  \cfoot[]{}
  \ofoot[]{\thepage}
  \addtokomafont{pageheadfoot}{\upshape}
}
\pagestyle{eisvogel-header-footer}

%%
%% end added
%%

%%%%%% Immer benötigte Packages
%
\usepackage[T1]{fontenc}		% sonst funktioniert die Silbentrennung bei Umlauten nicht
\usepackage[utf8]{inputenc}	% Eingabedekodierung. Ermöglicht Umlaute. Achtung: Unbedingt mit Betreuer
				% Verwendung der Umlaute-Eingabemethode absprechen. Im Zweifel \"O für Ö
\usepackage[ngerman]{babel}	% Silbentrennung und Sprachanpassung
\usepackage{blindtext}		% Blindtext
%\usepackage[hidelinks]{hyperref}		% Sprungmarken, z.B. im Inhaltsverzeichnis auf Textpassagen
\usepackage{graphicx}		% Definiert o.a. \includegraphics
\usepackage{textcomp}		% Sonst funktioniert z.B. \texteuro nicht
\usepackage{scrlayer-scrpage}	% Package zum Definieren der Kopf- und Fußzeilen
\usepackage{amsmath}		% Muss sein
\usepackage{mathrsfs} 	% Weitere Mathematik-Symbole, z.B. Laplace-L
%
%%%%% Anpassung an Formatvorlagen des Fachbereichs
%
\usepackage{helvet}		% Serifenlose Schrift ähnlich Arial
\renewcommand{\familydefault}{\sfdefault}	% als Standardschrift serifenlose Schrift verwenden
%
\usepackage{geometry} 		% Ränder direkt einstellen
\geometry{a4paper, top=20mm, left=30mm, right=20mm, bottom=25mm} % nach Vorgabe
\linespread{1.25} 		% Zeilenabstand nach Vorgabe
%
\usepackage{chngcntr}		% Ändert Verhalten von Countern
\counterwithout{figure}{section}	% Figure-Nummerierung nicht bei section-Wechsel zurücksetzen
\renewcommand{\thefigure}{\textbf\thechapter-\arabic{figure}}	% im Stil 3-2
%
%%%% für das Erzeugen von Grafiken mit Zeichenbefehlen
%
\usepackage{tikz}		% Grundpaket
\usetikzlibrary{shapes,arrows}	% einige Symbolpackages
\usepackage{tikz-cd}		% einige Symbolpackages


%%MOD%%
%
%%%%%%%%%%  Angaben für Titelseite %%%%%%%%%%%%%%%%%%%%%%
%
\input{thesis_title}

\begin{document}

%%MOD%%
% Einige Anpassungen müssen nach \begin{document} stehen !!
\renewcaptionname{ngerman}{\figurename}{\textbf Bild} 	% Bild ... statt Abbildung ... 
\renewcaptionname{ngerman}{\contentsname}{Inhalt}% Inhalt statt Inhaltsverzeichnis
%%%%%%%%%%%%%%%%%%%%%%%%%%%%%%%%%%%%%%%%%%%%%%%%%%%%%%%%%%%%
% Titel im FH Style
%%%%%%%%%%%%%%%%%%%%%%%%%%%%%%%%%%%%%%%%%%%%%%%%%%%%%%%%%%%%
\fhacmbtitle{\includegraphics[height=4cm]{fh_logo}}{5pt}{5pt}
%%%%%%%%%%%%%%%%%%%%%%%%%%%%%%%%%%%%%%%%%%%%%%%%%%%%%%%%%%%%
% Erklärung / Geheimhaltung
%%%%%%%%%%%%%%%%%%%%%%%%%%%%%%%%%%%%%%%%%%%%%%%%%%%%%%%%%%%%
%\frontmatter 	% Wenn der Hauptteil mit Seite 1 beginnen soll
\pagestyle{plain}
% \input{erklaerung}

%%%%%%%%%%%%%%%%%%%%%%%%%%%%%%%%%%%%%%%%%%%%%%%%%%%%%%%%%%%%
% Erklaerung
%%%%%%%%%%%%%%%%%%%%%%%%%%%%%%%%%%%%%%%%%%%%%%%%%%%%%%%%%%%%
\newpage

\newpage

%%%%%%%%%%%%%%%%%%%%%%%%%%%%%%%%%%%%%%%%%%%%%%%%%%%%%%%%%%%%
% Inhaltsverzeichnis
%%%%%%%%%%%%%%%%%%%%%%%%%%%%%%%%%%%%%%%%%%%%%%%%%%%%%%%%%%%%
\newpage
\tableofcontents
\newpage

%%%%%%%%%%%%%%%%%%%%%%%%%%%%%%%%%%%%%%%%%%%%%%%%%%%%%%%%%%%%
% CONTENT
%%%%%%%%%%%%%%%%%%%%%%%%%%%%%%%%%%%%%%%%%%%%%%%%%%%%%%%%%%%%
\pagestyle{scrheadings}
%%%%%%%%%%%%%%% Anpassung des Seitenstils an FH-Layoutvorschrift %%%%%%%%%%%%
%\renewcommand{\chaptermark}[1]{\markboth{\thechapter\hspace{1cm}#1}{}}	% Kapitel für Headerzeile neu definieren (ohne Nummer) TODO
\chead{}		% Header Mitte löschen
\ihead{\leftmark}	% Kapitelbezeichnung links setzen 
\renewcommand{\headfont}{\bfseries}	% Bold-Font für Headerzeile verwenden
\setheadsepline{0.5pt}



\hypertarget{vinaque-sanguine-metuenti-cuiquam-alcyone-fixus}{%
\section{Vinaque sanguine metuenti cuiquam Alcyone
fixus}\label{vinaque-sanguine-metuenti-cuiquam-alcyone-fixus}}

\hypertarget{aesculeae-domus-vincemur-et-veneris-adsuetus-lapsum}{%
\subsection{Aesculeae domus vincemur et Veneris adsuetus
lapsum}\label{aesculeae-domus-vincemur-et-veneris-adsuetus-lapsum}}

Lorem markdownum Letoia, et alios: figurae flectentem annis aliquid
Peneosque ab esse, obstat gravitate. Obscura atque coniuge, per de
coniunx, sibi \textbf{medias commentaque virgine} anima tamen comitemque
petis, sed. In Amphion vestros hamos ire arceor mandere spicula, in
licet aliquando.

\begin{lstlisting}[language=Java]
//JAVA CODE
public class Example implements LoremIpsum {
    public static void main(String[] args) {
        if(args.length < 2) {
            System.out.println("Lorem ipsum dolor sit amet");
        }
    } // Obscura atque coniuge, per de coniunx
}
\end{lstlisting}

\begin{lstlisting}[language={C++}]
//C++ CODE

#include <iostream>

int main() {
	bool test = 0;
    std::cout << "Hello World!";
    return 0;
}

}
\end{lstlisting}

\begin{lstlisting}[language={JSON}]
{
    "glossary": {
        "title": "example glossary",
		"GlossDiv": {
            "title": "S",
			"GlossList": {
                "GlossEntry": {
                    "ID": "SGML",
					"SortAs": "SGML",
					"GlossTerm": "Standard Generalized Markup Language",
					"Acronym": "SGML",
					"Abbrev": "ISO 8879:1986",
					"GlossDef": {
                        "para": "A meta-markup language, used to create markup languages such as DocBook.",
						"GlossSeeAlso": ["GML", "XML"]
                    },
					"GlossSee": "markup"
                }
            }
        }
    }
}

\end{lstlisting}




\begin{lstlisting}[language={QML}]
//QML CODE
Button {
            id: hb_button
            x: 731
            y: 4
            width: 61
            height: 61
            text: qsTr("BUTTON NAME")
            //REGISTER EVENT HANDLERS
            Connections {
                target: hb_settings_button      //FOR WHICH ELEMENT ID IS THE CONNECTION             
                function onClicked(_mouse){     //EVENT HANDLER FOR THE CLICK EVENT
                    hb_label.visible = true     //MODIFY OTHER QML ELEMENTS
                    main_menu.lb_settings_btn() //CALL A C++ BACKEND FUNCTION (main_menu is the QML instance of the C++ backend)
                }
            }
        }
\end{lstlisting}



\begin{lstlisting}[language={JavaScript}]
//JavaScript
function xmlToString(xmlData) {
    var xmlString;
    //IE
    if (window.ActiveXObject) {
        xmlString = xmlData.xml;
    }
    // code for Mozilla, Firefox, Opera, etc.
    else {
        xmlString = (new XMLSerializer()).serializeToString(xmlData);
    }
    return xmlString;
}

function __helper__svg_xml_to_string(_str){

            if(__helper__isArray(_str) || __helper__isHTMLCollection(_str)){
                var tmp_result = "";
                for(var i =0; i < _str.length;i++){
                    tmp_result += __helper__svg_xml_to_string(_str[i]);
                }
                return tmp_result;
            }

            return new XMLSerializer().serializeToString(_str);
        }







function IsJsonString(str) {
    try {
        JSON.parse(str);
    } catch (e) {
        return false;
    }
    return true;
}

function __helper__isArray(what) {
    return Object.prototype.toString.call(what) === '[object Array]';
}

function __helper__isHTMLCollection(what){
    return Object.prototype.toString.call(what) === '[object HTMLCollection]';

}


function __helper__fixedEncodeURIComponent (str) {
  return encodeURIComponent(str).replace(/[!'()*]/g, function(c) {
    return '%' + c.charCodeAt(0).toString(16);
  });
}
\end{lstlisting}


Porrigitur et Pallas nuper longusque cratere habuisse sepulcro pectore
fertur. Laudat ille auditi; vertitur iura tum nepotis causa; motus. Diva
virtus! Acrota destruitis vos iubet quo et classis excessere Scyrumve
spiro subitusque mente Pirithoi abstulit, lapides.

\hypertarget{lydia-caelo-recenti-haerebat-lacerum-ratae-at}{%
\subsection{Lydia caelo recenti haerebat lacerum ratae
at}\label{lydia-caelo-recenti-haerebat-lacerum-ratae-at}}

Te concepit pollice fugit vias alumno \textbf{oras} quam potest
\href{http://example.com\#rursus}{rursus} optat. Non evadere orbem
equorum, spatiis, vel pede inter si.

\begin{enumerate}
\def\labelenumi{\arabic{enumi}.}
\tightlist
\item
  De neque iura aquis
\item
  Frangitur gaudia mihi eo umor terrae quos
\item
  Recens diffudit ille tantum
\end{enumerate}

\begin{equation}\label{eq:neighbor-propability}
    p_{ij}(t) = \frac{\ell_j(t) - \ell_i(t)}{\sum_{k \in N_i(t)}^{} \ell_k(t) - \ell_i(t)}
\end{equation}

Tamen condeturque saxa Pallorque num et ferarum promittis inveni lilia
iuvencae adessent arbor. Florente perque at condeturque saxa et ferarum
promittis tendebat. Armos nisi obortas refugit me.

\begin{quote}
Et nepotes poterat, se qui. Euntem ego pater desuetaque aethera
Maeandri, et \href{http://example.com\#Dardanio_geminaque}{Dardanio
geminaque} cernit. Lassaque poenas nec, manifesta \(\pi r^2\) mirantia
captivarum prohibebant scelerato gradus unusque dura.
\end{quote}

\begin{itemize}
\tightlist
\item
  Permulcens flebile simul
\item
  Iura tum nepotis causa motus diva virtus Acrota. Tamen condeturque
  saxa Pallorque num et ferarum promittis inveni lilia iuvencae adessent
  arbor. Florente perque at ire arcum.
\end{itemize}

\end{document}
